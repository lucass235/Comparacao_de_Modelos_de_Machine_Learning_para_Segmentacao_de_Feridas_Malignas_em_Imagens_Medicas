\section{CRONOGRAMA SEMANAL}

A Tabela \ref{tab:cronograma} ilustra o cronograma proposto para o projeto, dividido em quinzenas, desde agosto até dezembro de 2023. As atividades listadas no cronograma incluem a revisão da literatura sobre arquiteturas de \ac{CNNs} para segmentação de feridas malignas, o desenvolvimento do modelo de \ac{CNNs}, o treinamento e a experimentação, avaliações, a revisão e a escrita do Trabalho de Conclusão de Curso (TCC), e a defesa do TCC. Este cronograma serve para orientar o progresso do trabalho e garantir que todas as etapas necessárias sejam concluídas em tempo hábil.

\begin{table}[htbp]
\centering
\caption{Cronograma de execução do Projeto de Pesquisa}
\label{tab:cronograma}
\begin{tblr}{
  row{1} = {c},
  cell{1}{2} = {c=9}{},
  cell{2}{2} = {c=2}{},
  cell{2}{4} = {c=2}{c},
  cell{2}{6} = {c=2}{c},
  cell{2}{8} = {c=2}{c},
  hlines,
  vlines,
}
\textbf{Atividades}               & \textbf{Quinzena} &   &              &   &              &   &              &   &              \\
                                  & \textbf{Ago }     &   & \textbf{Set} &   & \textbf{Out} &   & \textbf{Nov} &   & \textbf{Dez} \\
Pré-processamento das Imagens & X                 & X & X            &   &              &   &              &   &              \\
Treinamento e Ajuste do Modelo &                  & X &     X      &  X &           X  &  X &              &   &              \\
Avaliação de Desempenho           &                   &   &             & X & X            &  X &            &   &              \\
Análise Comparativa das Métricas  &                   &   &              & X & X            & X &              &   &              \\
Coleta das Métricas e Resultados  &                   &   &              &   &              & X & X            &   &              \\
Revisão/Escrita do TCC            &                   &  &  X            &  X &             X &  X & X            & X &              \\
Defesa do TCC                     &                   &   &              &   &              &   &              &   & X            
\end{tblr}
\end{table}