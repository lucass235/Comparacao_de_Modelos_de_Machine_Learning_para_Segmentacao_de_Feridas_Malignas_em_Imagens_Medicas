\section{FUNDAMENTAÇÃO TEÓRICA}

\subsection{Feridas Crônicas e Malignas}
%Definição e Classificação de Feridas Crônicas
%Epidemiologia e Impacto das Feridas Malignas
%Desafios no Tratamento de Feridas Malignas
%Qualidade de Vida e Aspectos Psicossociais Associados às Feridas Malignas

Feridas crônicas e malignas constituem um desafio notável na estomaterapia(Especialidade da enfermagem que se dedica ao cuidado de pacientes com estomas, feridas agudas ou crônicas, incontinência fecal ou urinária, e outras condições relacionadas ao trato gastrointestinal, urogenital e integumentar), exigindo uma compreensão profunda de suas características patológicas e implicações clínicas.

Feridas crônicas são definidas pela sua persistência além do tempo de cicatrização esperado, frequentemente excedendo três meses. A fisiopatologia subjacente é complexa, envolvendo inflamação crônica, angiogênese prejudicada e imunossupressão local. Estas características diferenciam as feridas crônicas das agudas e exigem uma abordagem terapêutica diferenciada.

As feridas malignas, um subconjunto específico de feridas crônicas, têm uma prevalência significativa, especialmente em populações com doenças crônicas subjacentes, como diabetes mellitus e imunossupressão. Elas representam um desafio tanto para o sistema de saúde quanto para os indivíduos afetados devido ao seu impacto na morbidade e mortalidade. A análise epidemiológica dessas feridas é fundamental para entender sua distribuição e fatores de risco.

No manejo de feridas malignas, a seleção de terapias apropriadas, incluindo agentes tópicos, terapias avançadas de cicatrização e abordagens cirúrgicas, é crítica. A gestão da dor e a prevenção de infecções secundárias são aspectos igualmente importantes. A terapia deve ser personalizada, considerando as características específicas da ferida e as condições do paciente.

As feridas malignas afetam profundamente a qualidade de vida dos pacientes, com implicações significativas na saúde mental, autoestima e capacidade funcional. Além do tratamento físico, é essencial abordar os aspectos psicológicos e sociais, fornecendo suporte emocional e psicossocial adequado.

A precisão na segmentação de feridas malignas em imagens médicas é crucial, não apenas para o diagnóstico e tratamento eficaz, mas também para a avaliação da progressão da doença e resposta ao tratamento. Esta precisão contribui significativamente para estratégias terapêuticas mais eficientes e personalizadas, impactando positivamente a recuperação e a qualidade de vida dos pacientes.

\subsection{Avaliação e Manejo de Feridas}
%Métodos Tradicionais de Avaliação de Feridas
%Parâmetros para Avaliação Clínica de Feridas
%Técnicas Avançadas de Tratamento de Feridas
%Diretrizes e Protocolos no Manejo de Feridas Malignas
%Tecnologias Emergentes no Tratamento de Feridas

A avaliação e manejo de feridas representam áreas cruciais na prática médica, abrangendo desde métodos tradicionais de observação visual até técnicas avançadas de tratamento. Ao entender a diversidade de técnicas abordados a seguir, torna-se possível compreender a complexidade envolvida no diagnóstico e tratamento de feridas, oferecendo insights valiosos para a prática médica contemporânea.

Métodos Tradicionais de Avaliação de Feridas: Os métodos tradicionais englobam técnicas que vão desde a observação visual até medições físicas. A inspeção visual é a forma mais básica de avaliação, onde profissionais de saúde examinam a aparência da ferida, observando características como tamanho, forma, cor, presença de tecido necrótico e exsudato. Além disso, a medição de áreas é feita para determinar a extensão da ferida, utilizando técnicas como planimetria e fotografia associada a software especializado. Esses métodos constituem a base da avaliação, permitindo o acompanhamento da progressão da ferida ao longo do tempo.

Parâmetros para Avaliação Clínica de Feridas: Durante a avaliação clínica, diversos parâmetros são considerados para determinar a condição e o progresso da ferida. Isso inclui a identificação da presença de infecção, analisando sinais como inflamação, calor, rubor, edema e dor. Além disso, a extensão da lesão é avaliada quanto à profundidade, dimensões e possíveis complicações, como formação de tecido de granulação e presença de bordas irregulares. A resposta ao tratamento também é monitorada, verificando se a ferida está cicatrizando de maneira eficaz.

Técnicas Avançadas de Tratamento de Feridas: As técnicas avançadas compreendem uma variedade de abordagens inovadoras para promover a cicatrização de feridas complexas. Isso inclui terapias tópicas, como o uso de substâncias bioativas, terapias de pressão negativa, que estimulam a circulação e removem fluidos da ferida, e terapias com oxigenação hiperbárica, que fornecem oxigênio em altas concentrações para acelerar a cicatrização. Essas técnicas representam um avanço considerável no manejo de feridas, oferecendo opções além dos métodos convencionais.

Diretrizes e Protocolos no Manejo de Feridas Malignas: Diretrizes e protocolos específicos são estabelecidos para lidar com feridas malignas, visando um diagnóstico precoce e um tratamento eficaz. Essas diretrizes definem critérios para identificar feridas malignas, estabelecem estratégias de tratamento e monitoramento, e fornecem orientações para o manejo de complicações, garantindo uma abordagem consistente e baseada em evidências.

Tecnologias Emergentes no Tratamento de Feridas: Tecnologias emergentes estão trazendo avanços significativos no tratamento de feridas, incluindo curativos inteligentes, que monitoram e reagem dinamicamente ao ambiente da ferida, e terapias de regeneração de tecidos, que estimulam o crescimento celular e a cicatrização de maneira mais eficiente. Além disso, há pesquisas em andamento para desenvolver abordagens personalizadas, utilizando impressão 3D para criar curativos adaptados às características únicas da ferida.

Esses pontos representam a diversidade e a complexidade das técnicas e abordagens disponíveis para a avaliação e tratamento de feridas. Desde métodos tradicionais até as mais avançadas tecnologias emergentes, o campo da medicina de feridas está constantemente evoluindo para oferecer soluções mais eficazes e personalizadas para os pacientes.

\subsection{Aplicações de IA no Diagnóstico e Tratamento Médico}
%Aplicações de IA no Diagnóstico e Tratamento Médico
%Desafios e Considerações Éticas da IA na Saúde
%Redes Neurais Convolucionais (CNNs)
%Desenvolvimento de Modelos de CNN para Segmentação de Feridas
%Comparação de Desempenho: Avaliação Manual vs. Automatizada
%Integração de CNNs na Prática Clínica
%Perspectivas Futuras e Potencial de Melhoria no Cuidado ao Paciente

A evolução da \ac{IA} revolucionou o cenário médico, proporcionando uma gama diversificada de aplicações no diagnóstico e tratamento. Neste contexto é explorado o impacto da IA na medicina, abordando desafios éticos, tecnologias como \ac{CNN} e seu desenvolvimento para segmentação de feridas. Também considera a comparação entre a avaliação manual e automatizada, bem como perspectivas futuras para aprimorar o cuidado ao paciente.

Desafios e Considerações Éticas da \ac{IA} na Saúde: Na aplicação da \ac{IA} na saúde, surgem desafios éticos significativos. A privacidade do paciente é crucial, exigindo medidas rigorosas para proteger e anonimizar os dados. Além disso, a interpretabilidade dos algoritmos de \ac{IA} é essencial para garantir que os profissionais possam compreender e confiar nas decisões tomadas pelos modelos. Questões de responsabilidade clínica também são pertinentes, pois os resultados gerados pela \ac{IA} influenciam as decisões dos profissionais de saúde.

Redes Neurais Convolucionais (CNNs): As \ac{CNNs} são um tipo de arquitetura de rede neural fundamental na análise de imagens médicas. Elas são projetadas para aprender padrões complexos e hierárquicos, identificando características e relações importantes entre os pixels das imagens. No contexto da segmentação de feridas malignas, as \ac{CNNs} são altamente eficazes na identificação e delimitação dessas regiões em imagens médicas.

Desenvolvimento de Modelos de \ac{CNN} para Segmentação de Feridas: Para criar um modelo de \ac{CNN} para a segmentação de feridas, é necessário coletar um conjunto de dados diversificado e anotado com precisão. Esse conjunto passa por etapas de pré-processamento para padronizar a qualidade das imagens e normalizar os dados. O treinamento do modelo envolve alimentar a \ac{CNN} com dados anotados para que ela possa aprender a identificar e segmentar as regiões de feridas malignas.

Comparação de Desempenho: Ao comparar a segmentação manual com a automatizada na detecção de feridas malignas em imagens médicas, é essencial considerar vários aspectos. A precisão e o recall, medindo a capacidade de identificação e abrangência das feridas, são vitais. Enquanto a segmentação manual é trabalhosa e subjetiva, a automação por IA oferece eficiência, consistência e potencial de generalização com base no treinamento. Contudo, a validação clínica é crucial para garantir que os resultados automatizados sejam clinicamente relevantes e confiáveis, alinhados às necessidades dos profissionais de saúde e dos pacientes.

Integração de \ac{CNNs} na Prática Clínica: A integração bem-sucedida de \ac{CNNs} na prática clínica envolve a validação dos modelos em ambientes clínicos reais. Isso inclui testes para verificar se os modelos são clinicamente relevantes, fáceis de interpretar e úteis para os profissionais de saúde, contribuindo para decisões mais precisas e eficazes.

Perspectivas Futuras e Potencial de Melhoria no Cuidado ao Paciente: As perspectivas futuras se concentram na contínua evolução da IA na área médica. Isso inclui aprimoramentos nos modelos de \ac{CNN}, a exploração de técnicas mais avançadas de aprendizado de máquina e a integração da IA em sistemas de saúde para um cuidado mais personalizado e preciso aos pacientes.

Esses pontos destacam a relevância da IA na segmentação de feridas malignas e evidenciam os desafios, considerações éticas e potenciais benefícios na aplicação dessas tecnologias na prática clínica.

\subsection{Arquiteturas de Redes Neurais}

As arquiteturas de redes neurais são estruturas organizadas de neurônios artificiais, modeladas com base no funcionamento do cérebro humano. No contexto da segmentação de imagens médicas, essas arquiteturas são adaptadas para analisar e compreender visualmente as informações contidas nessas imagens.

As \ac{CNNs} são altamente eficazes na análise de imagens devido à sua capacidade de preservar a relação espacial entre os pixels. Elas usam camadas convolucionais para extrair características hierárquicas das imagens, seguidas por camadas de pooling para reduzir a dimensionalidade e camadas totalmente conectadas para classificação ou segmentação.

Redes Neurais Recorrentes (RNNs): As RNNs são ideais para lidar com sequências de dados, como texto ou séries temporais médicas. Sua estrutura permite que informações sejam lembradas e aplicadas em etapas posteriores, sendo úteis em tarefas de previsão e análise temporal.

Redes Generativas Adversariais (GANs): As GANs consistem em duas redes neurais, um gerador e um discriminador, que competem entre si. Elas são usadas para criar novos dados realistas a partir de um conjunto de dados existente, sendo aplicáveis na geração de imagens médicas sintéticas para treinamento de modelos ou na correção/aperfeiçoamento de imagens existentes.

Fully Convolutional Network (\ac{FCN}): As \ac{FCN}s são arquiteturas especialmente concebidas para tarefas de segmentação e localização em imagens. Diferentemente das redes convolucionais convencionais, as \ac{FCN}s mantêm a estrutura completa das redes neurais convolucionais, porém substituem as camadas totalmente conectadas por convoluções globais, permitindo que a rede seja aplicada a imagens de qualquer tamanho.

\ac{U-Net}: A \ac{U-Net} é conhecida por sua eficácia em tarefas de segmentação semântica em imagens médicas. Sua arquitetura se assemelha à letra "U", com uma estrutura de codificação para a extração de características e uma estrutura de decodificação para reconstruir a imagem segmentada. Ela também incorpora conexões residuais entre camadas correspondentes, permitindo a preservação de detalhes durante a reconstrução.

\ac{SegNet}: A \ac{SegNet} é uma arquitetura de segmentação de imagens que se concentra na eficiência computacional. Ela utiliza um codificador convolucional para extrair características e um decodificador que mapeia características para a imagem segmentada. Seu destaque é o uso de mapas de índice durante a etapa de decodificação, ajudando na reconstrução da imagem.

\ac{MobileNetV2}: Esse modelo foi projetado para ser mais leve e eficiente, ideal para aplicativos móveis e tarefas em dispositivos com recursos limitados. Ele utiliza operações de convolução separável em largura e em profundidade para reduzir a complexidade computacional, mantendo um bom desempenho em tarefas de visão computacional, embora possa não ser ideal para segmentação detalhada de imagens médicas devido à sua arquitetura mais simplificada.

Esse conhecimento arquitetural é fundamental para entender o funcionamento dos modelos de Machine Learning utilizados no estudo. Ao explorar as técnicas de rede neural, pudemos avaliar a eficiência e aplicabilidade de diferentes modelos de aprendizado profundo, como \ac{U-Net}, \ac{SegNet}, \ac{FCN} e \ac{MobileNetV2}, na identificação rápida e precisa de feridas malignas. Isso poderá resultar em tratamentos mais eficientes e prognósticos mais positivos para os pacientes.

\subsection{Métricas}

No contexto de segmentação de imagens, as métricas são usadas para avaliar o quão bem o modelo está segmentando as áreas de interesse na imagem, como as áreas de feridas malignas. As métricas podem ser usadas para avaliar diferentes aspectos do desempenho do modelo, como precisão, completude, acurácia e similaridade com a segmentação manual. Ao avaliar as métricas, os pesquisadores podem determinar quais modelos são mais eficazes na segmentação de feridas malignas e ajudar a melhorar a precisão e eficiência dos modelos de Machine Learning. As métricas utilizadas neste estudo foram Loss, Precision, Recall e Coeficiente de Dice. 

Loss(Perda): É uma medida que quantifica o erro entre a segmentação produzida pelo modelo e a segmentação esperada. Geralmente, ela é calculada durante o treinamento do modelo, ajudando a ajustar os pesos da rede neural para minimizar esse erro. A redução da Loss indica uma melhor adaptação do modelo aos dados de treinamento.

Precision (Precisão): Essa métrica mede a proporção de pixels corretamente classificados como feridas malignas em relação ao total de pixels identificados pelo modelo como feridas malignas. Ela destaca a capacidade do modelo em não classificar incorretamente pixels saudáveis como feridas malignas.

Recall (Revocação): Refere-se à proporção de pixels de feridas malignas corretamente identificados pelo modelo em relação ao total de pixels de feridas malignas na imagem. Essa métrica destaca a habilidade do modelo em identificar corretamente a área das feridas malignas.

O Coeficiente de Dice: É uma métrica de similaridade que compara a segmentação produzida pelo modelo com a segmentação manual. Quanto mais próxima de 1, maior a sobreposição entre as duas segmentações. É especialmente útil quando há desequilíbrio entre as classes, como uma pequena quantidade de pixels de feridas malignas em relação ao total da imagem.