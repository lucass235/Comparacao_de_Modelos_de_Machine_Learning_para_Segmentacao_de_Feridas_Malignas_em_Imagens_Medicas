\section{FUNDAMENTAÇÃO TEÓRICA}

\subsection{Feridas Crônicas e Malignas}
%Definição e Classificação de Feridas Crônicas
%Epidemiologia e Impacto das Feridas Malignas
%Desafios no Tratamento de Feridas Malignas
%Qualidade de Vida e Aspectos Psicossociais Associados às Feridas Malignas

Feridas crônicas e malignas constituem um desafio notável na estomaterapia\footnote{Especialidade da enfermagem que se dedica ao cuidado de pacientes com estomas, feridas agudas ou crônicas, incontinência fecal ou urinária, e outras condições relacionadas ao trato gastrointestinal, urogenital e integumentar}, exigindo uma compreensão profunda de suas características patológicas e implicações clínicas.\cite{firmino2021topical}

\begin{itemize}
         
    \item As feridas crónicas são caracterizadas pela sua presença contínua para além do período típico de cicatrização, que muitas vezes é superior a três meses. A causa subjacente é multifacetada e envolve inflamação crônica, vasos sanguíneos prejudicados e supressão imunológica local. Esses atributos distinguem feridas crônicas de lesões agudas e necessitam de uma abordagem diferente para o tratamento.

    \item As feridas malignas, um tipo específico de ferida crónica que é particularmente prevalente em populações com diabetes ou com incapacidade de produzir anticorpos, são particularmente afetadas por esta doença. Representam uma ameaça tanto para o sistema de saúde como para os próprios indivíduos devido ao impacto que têm na doença e na morte. A investigação epidemiológica dessas lesões é crucial para a compreensão de sua prevalência e causas.

    \item A escolha de um regime de tratamento apropriado, incluindo agentes tópicos, terapias avançadas de cicatrização de feridas e técnicas cirúrgicas, é importante no tratamento de feridas malignas. O alívio da dor e a prevenção de infecções secundárias são igualmente importantes. O tratamento deve ser individualizado, levando em consideração as características específicas da ferida e o estado do paciente.

    \item Lesões violentas têm um impacto significativo na qualidade de vida do paciente, impactando significativamente sua saúde mental, autoestima e capacidade funcional. Além do tratamento físico, é importante abordar os aspectos psicológicos e sociais, fornecendo apoio emocional e psicossocial adequado.

    \item A precisão da segmentação de lesões malignas em imagens médicas é importante não apenas para diagnóstico e tratamento eficazes, mas também para avaliar a progressão da doença e a resposta ao tratamento. Essa precisão contribui muito para estratégias de tratamento mais eficazes e personalizadas, impactando positivamente na recuperação e na qualidade de vida do paciente.
    
\end{itemize}

\subsection{Avaliação e Manejo de Feridas}
%Métodos Tradicionais de Avaliação de Feridas
%Parâmetros para Avaliação Clínica de Feridas
%Técnicas Avançadas de Tratamento de Feridas
%Diretrizes e Protocolos no Manejo de Feridas Malignas
%Tecnologias Emergentes no Tratamento de Feridas

A avaliação e manejo de feridas representam áreas cruciais na prática médica, abrangendo desde métodos tradicionais de observação visual até técnicas avançadas de tratamento. Ao entender a diversidade de técnicas abordados a seguir, torna-se possível compreender a complexidade envolvida no diagnóstico e tratamento de feridas, oferecendo insights valiosos para a prática médica contemporânea. \cite{tsichlakidou2019intervention}

\begin{itemize} 

    \item Métodos tradicionais de avaliação de feridas: Os métodos tradicionais incluem métodos que vão desde a observação visual até medições físicas. O exame visual é a forma mais simples de avaliação, no qual um profissional médico examina a ferida em busca de  características como tamanho, formato, cor, presença de tecido necrótico e exsudato. Além disso, métodos como planimetria e fotografia usando ferramentas de software especificas são utilizados para medir a área e determinar a extensão da ferida. Este método é a base para uma avaliação que permite acompanhar a evolução da ferida ao longo do tempo.

    \item Parâmetros para Avaliação Clínica de Feridas: Durante a avaliação clínica, diversos parâmetros são considerados para determinar a condição e o progresso da ferida. Isso inclui a identificação da presença de infecção, analisando sinais como inflamação, calor, rubor, edema e dor. Além disso, a extensão da lesão é avaliada quanto à profundidade, dimensões e possíveis complicações, como formação de tecido de granulação e presença de bordas irregulares. A resposta ao tratamento também é monitorada, verificando se a ferida está cicatrizando de maneira eficaz.

    \item Técnicos avançadas de tratamento de feridas: As técnicos avançadas incluem uma variedade de abordagens inovadoras para promover a cicatrização de feridas complexos. Isto inclui tratamentos tópicos, como substâncias biologicamente ativas. terapia de pressão negativa, que estimula a circulação e remove fluido das feridas e oxigenoterapia hiperbárica que fornece uma alta concentração de oxigênio para promover a cura. Essas técnicos representam um avanço significativo no tratamento de feridas. Ele fornece recursos superiores aos métodos tradicionais.

    \item Diretrizes e Protocolos no Manejo de Feridas Malignas: Diretrizes e protocolos específicos são estabelecidos para lidar com feridas malignas, visando um diagnóstico precoce e um tratamento eficaz. Essas diretrizes definem critérios para identificar feridas malignas, estabelecem estratégias de tratamento e monitoramento, e fornecem orientações para o manejo de complicações, garantindo uma abordagem consistente e baseada em evidências.

    \item Tecnologias Emergentes no Tratamento de Feridas: Tecnologias emergentes estão trazendo avanços significativos no tratamento de feridas, incluindo curativos inteligentes, que monitoram e reagem dinamicamente ao ambiente da ferida, e terapias de regeneração de tecidos, que estimulam o crescimento celular e a cicatrização de maneira mais eficiente. Além disso, há pesquisas em andamento para desenvolver abordagens personalizadas, utilizando impressão 3D para criar curativos adaptados às características únicas da ferida.
    
\end{itemize}

    Esses pontos representam a diversidade e a complexidade das técnicas e abordagens disponíveis para a avaliação e tratamento de feridas. Desde métodos tradicionais até as mais avançadas tecnologias emergentes, o campo da medicina de feridas está constantemente evoluindo para oferecer soluções mais eficazes e personalizadas para os pacientes.

\subsection{Aplicações de IA no Tratamento Médico}

Os progressos no campo da \ac{IA} revolucionaram os cenários médicos, oferecendo diversas utilizações em diagnose e tratamento. Neste contexto, o impacto da \ac{IA} na medicina é explorado, abordando desafios éticos, tecnologias como a \ac{CNN} e o seu desenvolvimento para segmentação de traumas. Também considera a comparação entre avaliação manual e automatizada e as perspectivas futuras para melhorar o atendimento ao paciente. \cite{wang2018interactive}

\begin{itemize}

    \item Desafios e considerações éticas relativas à \ac{IA} nos cuidados de saúde: Sérias questões éticas emergem quando se aplica a \ac{IA} nos cuidados de saúde. A privacidade do paciente é tão importante que são necessários medidas rigorosas para proteger e anonimizar os dados. Além disso, a interpretabilidade dos algoritmos \ac{IA} é crucial para garantir que os especialistas possam comprender e confiar nas decisões do modelo. As questões de responsabilidade clínica também são relevantes, pois os resultados gerados pela \ac{IA} influenciarão as decisões dos profissionais de saúde.

    \item \ac{CNNs}: Este é um tipo de arquitetura de rede neural que é a base da análise de imagens médicas. Ele foi projetado para aprender padrões hierárquicos complexos, identificando características e relacionamentos importantes entre os pixels da imagem. No contexto da segmentação de feridas malignos, as \ac{CNNs} são extremamente eficazes na identificação e delineamento dessas regiões em imagens médicas.

    \item Desenvolvendo modelos \ac{CNN} para segmentação de feridas: Para criar um modelo \ac{CNN} para segmentação de feridas, você precisa coletar um conjunto de dados diversificado e cuidadosamente anotado. Este conjunto trespassa por uma etapa de pré-processamento para padronizar a qualidade da imagem e normalizar os dados. Treinar o modelo envolve alimentar \ac{CNN} com dados anotados para aprender a identificar e segmentar regiões de lesões malignos.

    \item Segmentação de feridas: A segmentação de feridas é uma técnica de processamento de imagens que visa identificar e rastrear a área afetada por uma ferida em imagens médicas. Essa técnica é importante porque permite que médicos e profissionais de saúde avaliar com acurácia o tamanho formato e localização das feridas, o que é fundamental para o correto diagnóstico e tratamento. Na prática a segmentação de feridas envolve a aplicação de algoritmos de aprendizagem profunda o imagens médicas de feridas. Esses algoritmos são treinados em um conjunto de dados de imagens rotuladas. Aqui está uma olhada no que os especialistas em trauma explicaram. Com base neste conjunto de dados, o algoritmo pode aprender a reconhecer características da ferida e segmentá-las em novas imagens.

    \item Comparação de desempenho Comparando segmentação manual e automática na segmentação de feridas cancerígenas em imagens médicas: É importante considerar vários aspectos. Exatidão e precisão, que medem a capacitância de identificar e cerrar uma ferida são importantes. Embora a segmentação manual seja trabalhosa e subjetiva, a automatização de CORRIA nesta tarefa de segmentação de imagens oferece eficiência, consistência e recursos de generalização baseados em treinamento. No entanto, a validação clínica é crucial para garantir que os resultados automatizados sejam clinicamente relevantes e confiáveis, alinhados com as necessidades dos profissionais de saúde e dos pacientes.

    \item Integração da \ac{CNN} na prática clínica: A integração bem-sucedida da \ac{CNNs} na prática clínica requer a validação de modelos em ambientes clínicos reais. Isto envolve testar se os modelos são clinicamente significativos, fáceis de interpretar e úteis para os profissionais de saúde, contribuindo assim para decisões mais precisos e eficazes.

    \item Perspectivas futuras e potencial para melhorar o atendimento ao paciente: As perspectivas futuras concentram-se na evolução contínua da IA na área médica. Isso inclui melhorias nos modelos \ac{CNN}, exploração de técnicos mais avançadas de aprendizado de máquina e integração de IA em sistemas de saúde para um atendimento ao paciente mais personalizado e preciso.
    
\end{itemize}

Estes pontos destacam a importância da inteligência artificial na segmentação de feridas malignos e destacam os desafios, as considerações éticas e os benefícios potenciais da aplicação destas tecnologias à prática clínica.

\subsection{Arquiteturas de Redes Neurais}

As arquiteturas de redes neurais são estruturas organizadas de neurônios artificiais, modeladas com base no funcionamento do cérebro humano. No contexto da segmentação de imagens médicas, essas arquiteturas são adaptadas para analisar e compreender visualmente as informações contidas nessas imagens. \cite{dean2022golden} \cite{jiang2023vig}

\begin{itemize}

    \item As \ac{CNNs} são altamente eficazes na análise de imagens devido à sua capacidade de preservar a relação espacial entre os pixels. Elas usam camadas convolucionais para extrair características hierárquicas das imagens, seguidas por camadas de pooling para reduzir a dimensionalidade e camadas totalmente conectadas para classificação ou segmentação.

    \item Redes Neurais Recorrentes (RNNs): As RNNs são ideais para lidar com sequências de dados, como texto ou séries temporais médicas. Sua estrutura permite que informações sejam lembradas e aplicadas em etapas posteriores, sendo úteis em tarefas de previsão e análise temporal.

    \item Redes Generativas Adversariais (GANs): As GANs consistem em duas redes neurais, um gerador e um discriminador, que competem entre si. Elas são usadas para criar novos dados realistas a partir de um conjunto de dados existente, sendo aplicáveis na geração de imagens médicas sintéticas para treinamento de modelos ou na correção/aperfeiçoamento de imagens existentes.

    \item Fully Convolutional Network (\ac{FCN}): As \ac{FCN}s são arquiteturas especialmente concebidas para tarefas de segmentação e localização em imagens. Diferentemente das redes convolucionais convencionais, as \ac{FCN}s mantêm a estrutura completa das redes neurais convolucionais, porém substituem as camadas totalmente conectadas por convoluções globais, permitindo que a rede seja aplicada a imagens de qualquer tamanho.

    \item \ac{U-Net}: A \ac{U-Net} é conhecida por sua eficácia em tarefas de segmentação semântica em imagens médicas. Sua arquitetura se assemelha à letra "U", com uma estrutura de codificação para a extração de características e uma estrutura de decodificação para reconstruir a imagem segmentada. Ela também incorpora conexões residuais entre camadas correspondentes, permitindo a preservação de detalhes durante a reconstrução.

    \item \ac{SegNet}: A \ac{SegNet} é uma arquitetura de segmentação de imagens que se concentra na eficiência computacional. Ela utiliza um codificador convolucional para extrair características e um decodificador que mapeia características para a imagem segmentada. Seu destaque é o uso de mapas de índice durante a etapa de decodificação, ajudando na reconstrução da imagem.

    \item \ac{MobileNetV2}: Esse modelo foi projetado para ser mais leve e eficiente, ideal para aplicativos móveis e tarefas em dispositivos com recursos limitados. Ele utiliza operações de convolução separável em largura e em profundidade para reduzir a complexidade computacional, mantendo um bom desempenho em tarefas de visão computacional, embora possa não ser ideal para segmentação detalhada de imagens médicas devido à sua arquitetura mais simplificada.

\end{itemize}

Esse conhecimento arquitetural é fundamental para entender o funcionamento dos modelos de Machine Learning utilizados no estudo. Ao explorar as técnicas de rede neural, pudemos avaliar a eficiência e aplicabilidade de diferentes modelos de aprendizado profundo, como \ac{U-Net}, \ac{SegNet}, \ac{FCN} e \ac{MobileNetV2}, na identificação rápida e precisa de feridas malignas. Isso poderá resultar em tratamentos mais eficientes e prognósticos mais positivos para os pacientes.

\subsection{Métricas}

No contexto de segmentação de imagens, as métricas são usadas para avaliar o quão bem o modelo está segmentando as áreas de interesse na imagem, como as áreas de feridas malignas. As métricas podem ser usadas para avaliar diferentes aspectos do desempenho do modelo, como precisão, completude, acurácia e similaridade com a segmentação manual. Ao avaliar as métricas, os pesquisadores podem determinar quais modelos são mais eficazes na segmentação de feridas malignas e ajudar a melhorar a precisão e eficiência dos modelos de Machine Learning. As métricas utilizadas neste estudo foram Loss, Precision, Recall e Coeficiente de Dice. \cite{kelleher2019deep} \cite{taha2015metrics} \cite{guan2022informing}

Loss (Perda): É uma medida que quantifica o erro entre a segmentação produzida pelo modelo e a segmentação esperada. Geralmente, ela é calculada durante o treinamento do modelo, ajudando a ajustar os pesos da rede neural para minimizar esse erro. A redução da Loss indica uma melhor adaptação do modelo aos dados de treinamento.

Precision (Precisão): Essa métrica mede a proporção de pixels corretamente classificados como feridas malignas em relação ao total de pixels identificados pelo modelo como feridas malignas. Ela destaca a capacidade do modelo em não classificar incorretamente pixels saudáveis como feridas malignas.

Recall (Revocação): Refere-se à proporção de pixels de feridas malignas corretamente identificados pelo modelo em relação ao total de pixels de feridas malignas na imagem. Essa métrica destaca a habilidade do modelo em identificar corretamente a área das feridas malignas.

O Coeficiente de Dice: É uma métrica de similaridade que compara a segmentação produzida pelo modelo com a segmentação manual. Quanto mais próxima de 1, maior a sobreposição entre as duas segmentações. É especialmente útil quando há desequilíbrio entre as classes, como uma pequena quantidade de pixels de feridas malignas em relação ao total da imagem.