\begin{center}
    \textbf{RESUMO}
\end{center}

\textbf{Contexto}: A área da saúde demanda cada vez mais a segmentação precisa de imagens médicas, notadamente no ramo da oncologia cutânea. Nesse contexto, a identificação exata e rápida de feridas malignas pode resultar em tratamentos mais eficientes e prognósticos mais positivos. Modelos de aprendizado profundo, como \acf{U-Net}, \acf{SegNet}, \acf{FCN} e \acf{MobileNetV2}, têm ganhado espaço nesse cenário devido à sua capacidade e potencial de aplicação. \textbf{Objetivo}: Este estudo visa explorar a eficiência e aplicabilidade desses modelos de aprendizado profundo no que tange à segmentação de feridas malignas em imagens médicas, levando em conta seus resultados no que diz a respeito a precisão do resultados dos algoritmos. \textbf{Método}: A metodologia adotada para essa investigação percorre uma série de fases, desde o pré-processamento das imagens até a avaliação do desempenho dos modelos de aprendizado profundo, passando por testes em diferentes modelos de machine learning. O desempenho de cada modelo foi avaliado por meio de métricas consagradas, como Loss, Precison, Recall e Coeficiente de Dice, além de ser considerada a eficiência computacional de cada um. \textbf{Resultados}: Os resultados obtidos neste estudo foram promissores, os modelos avaliados demonstraram alto desempenho na segmentação de feridas malignas e forneçeram insights significativos a respeito do desempenho comparativo entre diferentes arquiteturas de aprendizado profundo em aplicações médicas. \textbf{Conclusão}: Espera-se que as descobertas deste estudo ofereçam direcionamentos para futuras pesquisas no campo da segmentação de imagens médicas via aprendizado profundo. Ademais, a pesquisa tem o potencial de trazer benefícios notáveis à medicina -- sobretudo à oncologia cutânea -- ao prover ferramentas automatizadas e eficazes para segmentação de feridas malignas, colaborando, assim, com diagnósticos e monitoramento de tratamentos por profissionais da saúde.

\textbf{Palavras-chave:}  Câncer; Feridas Malignas; Segmentação de Feridas; Redes Neurais Convolucionais.