\section{METODOLOGIA}

\subsection{\red{Método}}

    Neste projeto de pesquisa foi empregado diversos modelos de aprendizagem profunda, tais como \ac{FCN}, \ac{U-Net}, \ac{SegNet} e \ac{MobileNetV2}, para segmentar feridas malignas em imagens médicas. Utilizaremos um grande conjunto de dados que reuniu uma junção de vários dataset de feridas. As imagens, que apresentam diversas formas e variações, serão pré-processadas antes do treinamento. Avaliaremos quantitativamente a performance dos modelos com base na área da ferida, precisão e eficiência do modelo na segmentação das imagens. A Figura~\ref{fig:diagrama} demonstra o fluxograma da metodologia aplicada.

    \begin{figure}[htbp]
        \centering
        \caption{Diagrama da Metodologia}
        \includegraphics[width=0.8\textwidth]{img/Diagrama.png}
        \label{fig:diagrama}
        \par\medskip\textbf{Fonte:} Autor
    \end{figure}

    \subsubsection{Conjunto de Dados}
        O conjunto de dados deste estudo compreende imagens de feridas malignas coletadas de diversas fontes, incluindo repositórios públicos no GitHub e sites especializados em imagens médicas. Foi selecionado mais de 4.800 imagens para treinar e testar os modelos de aprendizado profundo, considerando a diversidade de tipos de feridas malignas e a qualidade das imagens. A utilização de um conjunto de dados amplo e diversificado contribuirá para aprimorar os resultados deste estudo e desenvolver modelos de aprendizado profundo mais eficazes na segmentação de feridas malignas.
        
    \subsubsection{Criação do Conjunto de Dados}
        
        \begin{itemize}
            \item Tipo de imagens: As imagens médicas incluídas neste conjunto abrangem diversos tipos, tais Feridas de úlceras em pé diabéticos, lesões com cortes profundas e feridas crônicas em diversas partes do corpo. Suas características, como resolução, dimensões e formato de arquivo, são especificadas para proporcionar uma compreensão detalhada. Detalhamos o processo de aquisição, incluindo informações sobre o equipamento utilizado, configurações e protocolos adotados para a captura dessas imagens médicas.
        
            \item Pré-processamento e Anotação: Descrevemos as técnicas de pré-processamento aplicadas, como normalização e aumento de dados, ressaltando a importância dessas etapas na preparação das imagens para análise. O processo de anotação é abordado, incluindo responsáveis e critérios utilizados. Foi abordado nos dados sobre a diversidade do conjunto, considerando variabilidade em condições médicas, faixas etárias, gêneros e outros fatores relevantes, garantindo representatividade.
        
            \item Volume de Dados: Informamos a quantidade total de imagens e casos incluídos no dataset, proporcionando uma visão abrangente de sua robustez e amplitude. Consideramos mais de 4.800 imagens no dataset de diferentes fontes públicas e de diferentes condições clínicas para atender a heterogeneidade dos dados.
        
            \item Questões Éticas e de Privacidade: Abordamos as questões éticas, incluindo o consentimento informado, os processos de anonimização de dados e a conformidade com regulamentos de privacidade e proteção de dados.
    
            \item Qualidade e Confiabilidade dos Dados: Sobre a qualidade das imagens, considerando resolução e clareza, e a confiabilidade das anotações. Destacamos qualquer validação realizada por especialistas médicos para assegurar a precisão.
    
            \item Disponibilidade e Acesso: Fornecemos informações sobre a disponibilidade pública do dataset, incluindo detalhes sobre como acessá-lo, bem como eventuais restrições ou requisitos associados.
    
            \item Potenciais Aplicações e Limitações: Descrevemos possíveis aplicações do conjunto de dados em modelos de visão computacional, destacando suas potencialidades. Além disso, discutimos abertamente quaisquer limitações conhecidas ou possíveis viéses que devem ser considerados durante o uso do dataset.
    
            
        \end{itemize}

\subsection{\red{Execução do Método}}      

    \subsubsection{Pré-processamento de Imagens}
    
        As imagens foram inicialmente redimensionadas para uma resolução de 256x256 pixels para normalizar o tamanho da imagem e facilita o processamento de acordo com vários modelos. Os valores de píxel foram então normalizados para um intervalo de 0 a 1 para garantir uma escala de intensidade consistente em todas as imagens.
    
        Além disso, técnicos de aumento de dados, como rotação, inversão horizontal e dimensionamento É usado para aumentar a diversidade de conjuntos de dados e evitar overfitting. Essas técnicos permitem que os modelos aprendem a reconhecer as características das lesões malignos em diferentes locais e escalas.
    
        Vale ressaltar que nenhum filtro de suavização foi aplicado nas imagens para preservar as características originais das feridas. Isto é crucial para garantir que os modelos aprendem a reconhecer as especificidades das lesões malignos e não sejam afetados por artefatos de imagem. Estas são fases críticas no treinamento de modelos de Machine Learning para identificar lesões malignos. Eles permitem que os modelos aprendem a reconhecer as especificidades das lesões malignos de forma mais flexível e precisa, resultando em segmentações mais precisos e confiáveis.
    
    \subsubsection{Divisão do Conjunto de Dados}
        \red{A divisão do conjunto de dados foi executada estratificadamente, garantindo uma distribuição uniforme das classes de feridas malignas em cada subconjunto. Essa abordagem é fundamental para manter a representatividade e o equilíbrio dos dados, mitigando possíveis viéses e reforçando a confiabilidade dos resultados.}
    
        \red{Para garantir robustez estatística, o conjunto de dados foi submetido a um processo estocástico de 10 a 15 repetições na divisão. Cada iteração foi conduzida de forma aleatória, mas estratificada, preservando as proporções de cada classe de feridas malignas em cada segmento. Essa abordagem sistemática permitiu avaliar a estabilidade e consistência dos modelos diante de diferentes divisões de dados, minimizando o impacto de flutuações aleatórias nos resultados.}
    
        \red{O conjunto foi dividido em dois grupos principais: treinamento e teste. O grupo de treinamento desempenhou um papel vital no aprendizado dos modelos de aprendizado profundo, enquanto o grupo de teste foi fundamental para avaliar a eficácia dos modelos em dados não vistos anteriormente. Essa metodologia, embora aleatória, foi cuidadosamente projetada para manter um ambiente de dados equilibrado e representativo em todas as iterações.}
    
        \red{Ao realizar múltiplas repetições estocásticas, asseguramos que tanto o treinamento quanto a avaliação dos modelos ocorram em ambientes de dados variados e balanceados. Isso não apenas fortaleceu a confiabilidade dos resultados, mas também ofereceu uma visão abrangente da capacidade de generalização dos modelos frente a diferentes cenários de divisão de dados.}
    
    \subsubsection{Arquiteturas de Redes Neurais}
        Para realizar a segmentação das feridas malignas cutâneas, foram exploradas quatro arquiteturas de redes neurais convolucionais. \red{A escolha das arquiteturas adotadas neste estudo foi cuidadosamente fundamentada em um levantamento abrangente da literatura. Trabalhos anteriores e pesquisas de referência na área de segmentação de imagens médicas destacaram consistentemente essas arquiteturas como os modelos mais utilizados e bem-sucedidos para essa tarefa específica. Essas arquiteturas ganharam destaque devido à sua eficácia em lidar com desafios complexos de segmentação de feridas malignas.}
    
        \clearpage

        \begin{itemize}
            
            
        \item {FCN}
        
            A arquitetura \ac{FCN}, representada abaixo na figura \ref{fig:arquiteturaFCN},  é conhecida por sua capacidade de realizar a segmentação semântica em imagens. Ela consiste em uma \ac{CNN} totalmente composta por camadas convolucionais, sem camadas totalmente conectadas. \red{\cite{long2015fully}}
    
            \begin{figure}[htbp]
                \centering
                \caption{Representação Esquemática da Arquitetura \ac{FCN}.}
                \includegraphics[width=0.8\textwidth]{img/arquitetura_FCN.png}
                \label{fig:arquiteturaFCN}
                \par\medskip\textbf{Fonte:} adaptada de (\cite{long2015fully})
            \end{figure}
            
            A Figura \ref{fig:arquiteturaFCN} acima, apresenta a arquitetura de uma rede \ac{FCN} utilizada para tarefa de segmentação semântica (semantic segmentation), isto é, classificar cada pixel da imagem de entrada de acordo com a classe que ele pertence, sendo: cama, pé ou ferida (background). Conforme a arquitetura apresentada na Figura, existem várias camadas de convolução que produzirão mapas de características de diferentes profundidades. No final da rede, encontra-se a previsão pixelwise (pixelwise prediction) que também é um tipo de camada de convolução e que irá fazer uma predição pixel-a-pixel, isto é, atribuindo cada pixel a uma respectiva classe. Esta representação ilustra de forma esquemática a arquitetura \ac{FCN}, mostrando as camadas convolucionais e suas dimensões. Essa arquitetura é capaz de extrair as características mais importantes das imagens de feridas malignas, permitindo que a rede aprenda a segmentar essas feridas com precisão. 
    
        \clearpage
        
        \item {U-Net}
    
            A arquitetura \ac{U-Net}, ilustrada na figura \ref{fig:arquiteturaUNet} abaixo,  é amplamente utilizada para tarefas de segmentação em imagens biomédicas. Ela possui uma estrutura em forma de U, com um encoder para capturar informações contextuais e um decoder para reconstruir a máscara de segmentação. A \ac{U-Net} é conhecida por sua capacidade de segmentação precisa e é aplicada com sucesso em diversos problemas de segmentação, incluindo a segmentação de feridas medicas. \red{\cite{ronneberger2015u}}
    
            \begin{figure}[htbp]
                \centering
                 \caption{Representação Esquemática da Arquitetura \ac{U-Net}. }
                \includegraphics[width=0.8\textwidth]{img/arquitetura_U-Net.png}
                \label{fig:arquiteturaUNet}
                \par\medskip\textbf{Fonte:} adaptada de (\cite{ronneberger2015u})
            \end{figure}
            
                A Figura \ref{fig:arquiteturaUNet} acima, ilustra a arquitetura da rede \ac{U-Net}, em que cada caixinha azul presente na imagem corresponde a um mapa de característica multicanal (multichannel feature map). O número de cada canal está descrito no valor acima de cada caixa. No canto inferior esquerdo é dada a dimensão x-y da imagem. As caixas brancas representam a cópia dos mapas de características (feature maps) e cada flecha com sua respectiva cor representa uma operação diferente. Na parte direita da rede as flechas verdes referem-se ao caminho de expansão onde é utilizado a operação de up-convolution, também chamada de de-convolution6 ou transposed convolution. A figura ilustra essa arquitetura de forma esquemática, mostrando as camadas convolucionais, as camadas de pooling máximo e up-sampling, e as conexões laterais entre as camadas do caminho de contração e do caminho de expansão.
    
    
        
        \item {SegNet}
    
            O modelo \ac{SegNet}, representado na figura \ref{fig:arquiteturaSegNet} abaixo,  é baseado em uma arquitetura de codificador-decodificador. Cada codificador aplica convolução, normalização de lote e uma não linearidade, e depois aplica um pool máximo no resultado, enquanto armazena o índice do valor extraído de cada janela. Os decodificadores são semelhantes aos codificadores, a diferença é que eles não têm uma não linearidade e aumentam a amostra de entrada, usando índices armazenados a partir do estágio de codificação.\red{\cite{badrinarayanan2017deep}}
    
            \begin{figure}[htbp]
                \centering
                \caption{Representação Esquemática da Arquitetura \ac{SegNet}.}
                \includegraphics[width=0.9\textwidth]{img/arquitetura_Seg-Net.png}
                \label{fig:arquiteturaSegNet}
                \par\medskip\textbf{Fonte:} adaptada de (\cite{badrinarayanan2017deep})
            \end{figure}
            
                A Figura \ref{fig:arquiteturaSegNet} acima, ilustra essa arquitetura de forma esquemática, mostrando as camadas de codificação e decodificação, bem como as conexões entre elas. Cada caixa na figura representa uma camada de convolução, normalização de lote e não linearidade, enquanto as setas representam as conexões entre as camadas. As camadas de pooling máximo são representadas pelas caixas de cor verde. Além disso, a figura também mostra a saída da rede, que é uma imagem segmentada com as áreas de feridas malignas destacadas em branco. Essa saída é gerada pela última camada de decodificação da rede.
                Em resumo, a Figura ilustra de forma esquemática a arquitetura \ac{SegNet}, mostrando as camadas de codificação e decodificação, bem como as conexões entre elas. Essa arquitetura é capaz de segmentar com precisão as feridas malignas em imagens médicas, como mostrado nos resultados do estudo.
    
        \clearpage
        
        \item {MobileNetV2}
    
            O \ac{MobileNetV2}, representada na figura \ref{fig:arquiteturaMobileNetV2} abaixo, é uma arquitetura de \ac{CNN} projetada para tarefas de classificação e segmentação em dispositivos com recursos computacionais limitados. Essa arquitetura utiliza camadas convolucionais separáveis em profundidade para obter um bom equilíbrio entre a precisão do modelo e a eficiência computacional. \red{\cite{akay2021deep}}
        
                \begin{figure}[htbp]
                    \centering
                    \caption{Representação Esquemática da Arquitetura \ac{MobileNetV2}.}
                    \includegraphics[width=0.8\textwidth]{img/arquitetura_MobileNetV2.png}
                    \label{fig:arquiteturaMobileNetV2}
                    \par\medskip\textbf{Fonte:} adaptada de (\cite{akay2021deep})
                \end{figure}   
        
            A figura \ref{fig:arquiteturaMobileNetV2} mostra esquematicamente esta arquitetura. Revela a camada de convolução e seu tamanho. A imagem de entrada é uma imagem de ferida com dimensões 128x128x3, que é processada pela primeira camada convolucional com dimensões 128x128 e um número de filtros (ou canais) igual a 32. Em seguida a imagem é processada por uma segunda camada convolucional com dimensões 64x64 e uma número de filtros igual a 32. Em seguida, a imagem é processada por diversas camadas convolucionais com dimensões 32x32 e 96 filtros, que são responsáveis por extrair características mais complexos da imagem Estas camadas são seguidas por uma camada convolucional de dimensões 4x4 e um número de filtros igual a 1280, responsáveis por extrair as características mais importantes da imagem Por fim, a saída da última camada convolucional é processada por uma rede totalmente conectada com número de neurônios igual a 1280, que é responsável por gerar a saída final da rede.
            
    \end{itemize}
    \subsubsection{Treinamento e Ajuste do Modelo}
    
        Durante o processo de treinamento, uma grande base de imagens de feridas foi usada para instruir os modelos sobre como realizar a segmentação precisa dessas regiões. Essa base de dados foi dividida em dois conjuntos distintos: um conjunto de treinamento e outro de teste. O conjunto de treinamento é empregado para ensinar o modelo, enquanto o conjunto de teste é utilizado para avaliar a capacidade de generalização do modelo para dados não vistos previamente.
    
        Foi aplicado técnicas de aumento de dados, uma prática que visa aumentar a diversidade e a quantidade do conjunto de treinamento. Isso é alcançado por meio de transformações como rotações, reflexões e ajustes nas imagens originais, o que proporciona ao modelo uma exposição a uma gama mais ampla de variações nas feridas malignas.
    
        Além disso, foi utilizado a técnica de poda nos modelos discutidos, visando reduzir a complexidade dos mesmos e melhorar sua eficiência computacional. A poda concentra-se na eliminação de parâmetros menos relevantes, mantendo apenas aqueles mais significativos para o desempenho do modelo. Isso mostra uma melhora na eficiência de processamento e no uso de memória do modelo, sem comprometer a precisão da segmentação.
    
        \red{A seleção adequada dos parâmetros de treinamento é um componente crítico para otimizar o desempenho dos modelos de segmentação de feridas malignas. Durante o treinamento, uma série de parâmetros é ajustada para garantir a convergência ideal do modelo e sua capacidade de generalização. A taxa de aprendizado, um dos parâmetros mais influentes, foi meticulosamente ajustada para controlar a magnitude das atualizações nos pesos do modelo. É usado também métodos de otimização, como busca em grade e otimização bayesiana, para determinar a taxa de aprendizado mais adequada, evitando convergência rápida demais ou estagnação durante o treinamento. O tamanho do lote foi escolhido considerando a capacidade de memória disponível e o impacto no desempenho do modelo. Uma cuidadosa análise foi realizada para equilibrar a eficiência computacional com a qualidade da convergência. O número de épocas foi determinado utilizando técnicas de validação cruzada e monitoramento do desempenho do modelo no conjunto de validação. Essa estratégia nos permitiu encontrar um equilíbrio entre a quantidade de iterações necessárias para a convergência e a prevenção do sobreajuste.}
    
        O processo de otimização dos modelos envolve a sintonia dos hiperparâmetros, como a taxa de aprendizado, o tamanho do lote e o número de épocas de treinamento. Essa etapa visa encontrar a combinação ideal de parâmetros que maximize a precisão da segmentação e evite problemas de overfitting, onde o modelo se ajusta excessivamente aos dados de treinamento, comprometendo sua capacidade de generalização para novos dados.
    
    \subsubsection{Avaliação e Métricas}
        Para avaliar a eficácia dos modelos de segmentação de lesões malignas em imagens médicas, aplicou-se um conjunto de métricas essenciais. Utilizou-se a métrica de Loss para quantificar a discrepância entre as segmentações previstas pelos modelos e as reais, com valores menores indicando maior precisão na segmentação. A métrica Dice, que avalia a sobreposição entre as previsões do modelo e a verdade padrão, é outra ferramenta crucial, onde resultados mais próximos de 1 representam uma sobreposição ideal. Precisão, que mede a exatidão do modelo na identificação correta das lesões malignas, esse métrica avalia a proporção de verdadeiros positivos frente às predições positivas. Essas métricas conjuntas proporcionam uma análise detalhada do desempenho, orientando ajustes e melhorias. Valores ideais são definidos conforme as demandas clínicas, assegurando a confiabilidade dos processos de segmentação. A extensão das lesões foi quantificada, e testes estatísticos t foram utilizados para discernir diferenças significativas entre os modelos. Essa metodologia abrangente garante uma avaliação precisa dos modelos de segmentação, crucial para a prática médica.

\subsection{Considerações Éticas}

    Dado as diretrizes das imagens médicas de pacientes, atendemos rigorosamente às considerações éticas. Anonimizaremos todas as imagens, removendo dados identificáveis para assegurar a privacidade dos pacientes. Este projeto caminhou para atender as diretrizes da Declaração de Helsinque para pesquisas envolvendo seres humanos. Essa declaração é um conjunto de princípios éticos que orientam a pesquisa médica envolvendo seres humanos.
    
    A implementação desta metodologia permitirá avaliar a eficácia de vários modelos de aprendizagem profunda na segmentação de feridas malignas em imagens médicas. Compararemos os modelos usando uma variedade de métricas para fornecer insights valiosos para o desenvolvimento de futuros sistemas de diagnóstico assistido por computador na área de oncologia cutânea.

