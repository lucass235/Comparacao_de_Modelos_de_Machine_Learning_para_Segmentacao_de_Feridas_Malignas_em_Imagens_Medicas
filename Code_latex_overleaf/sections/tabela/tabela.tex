\begin{table}
\centering
\caption{Habilidades necessárias para se trabalhar com computação em nuvem}
\begin{tabular}{|l|l|ll} 
\cline{1-2}
\textbf{ Habilidades }                                                                                                              & \textbf{ Especificação da Habilidade }                                                                                                                                                                                                                                                                                                                                                                                                                                                                                                                                                                                                                  &  &   \\ 
\cline{1-2}
\begin{tabular}[c]{@{}l@{}}Linguagens de programação Perl, \\Ruby, Ruby on Rails e Python, \\além de Java e JavaScript\end{tabular} & \begin{tabular}[c]{@{}l@{}}Todos os projetos envolvendo \\tecnologia em nuvem\\~utilizam essas linguagens. \\É desejável que os \\profissionais da área saibam\\~pelo menos um das\\~listadas e quanto mais, \\melhor.\end{tabular}                                                                                                                                                                                                                                                                                                                                                                                                                     &  &   \\ 
\cline{1-2}
Conceito DevOps                                                                                                                     & \begin{tabular}[c]{@{}l@{}}A Computação em nuvem \\requer que as pessoas\\~entendam o trabalho \\em rede, sua infraestrutura \\e seu desenvolvimento, \\assim como suas operações.\\~Com o DevOps fica mais \\fácil de compreender as \\várias peças da TI e como \\elas funcionam para os negócios.\end{tabular}                                                                                                                                                                                                                                                                                                                                       &  &   \\ 
\cline{1-2}
\begin{tabular}[c]{@{}l@{}}Linguagens de programação \\SQL e MySQL\end{tabular}                                                     & \begin{tabular}[c]{@{}l@{}}Faz-se necessário compreender \\como funciona uma\\~base de dados. E não \\somente as duas linguagens básicas,\\~mas também outras plataformas, como \\o software para processamento \\de dados Hadoop; a Cassandra,\\que oferece sistema de gerenciamento de \\banco de dados; e a MongoDB,\\~aplicação para documentos de \\um banco de dados, todos os\\três em código aberto. Isso ajuda \\os profissionais a entenderem \\o que é essencial no Computação\\~em nuvem, como as pessoas \\acessam conteúdo e como isso vai \\e volta no núcleo de seus trabalhos.\end{tabular}                                            &  &   \\ 
\cline{1-2}
\begin{tabular}[c]{@{}l@{}}Desenvolvimento para \\aplicativos móveis\end{tabular}                                                   & \begin{tabular}[c]{@{}l@{}}O mundo está utilizando cada \\vez mais dispositivos móveis. \\Compreender como a \\tecnologia em nuvem \\pode funcionar \\com os aplicativos para \\esses aparelhos é essencial, \\principalmente quando há\\~necessidade de unir \\características específicas \\de cada área.\end{tabular}                                                                                                                                                                                                                                                                                                                                &  &   \\ 

\cline{1-2}
\end{tabular}
\end{table}
%-------------------------------------
\begin{table}
\centering
\caption{Habilidades necessárias para se trabalhar com computação em nuvem}
\begin{tabular}{|l|l|} 


\hline
Conceito
  de virtualização           & \begin{tabular}[c]{@{}l@{}}Todo mundo que trabalha \\na área de TI, em geral, \\precisa compreender o \\que é virtualização da tecnologia,\\que, basicamente, é a \\simulação de hardwares e softwares\\~reais em ambiente virtual. \\Não é assim tão simples, claro,\\e conhecer o conceito, \\sua extensão e aplicação são \\primordiais para trabalhar\\~com Computação em nuvem.\end{tabular}                                                                                                                                     \\ 
\hline
Especialidade em sistemas específicos & \begin{tabular}[c]{@{}l@{}}A maioria das companhias \\querem profissionais \\que saibam lidar com pelo \\menos as plataformas \\em nuvem do Google ou \\da A\textit{mazon.} Mas também \\precisam de gente que \\entenda de sistemas específicos, \\como os que utilizam o \\“Software as a Service” \\(software como serviço, ou SaaS), \\a exemplo da\textit{ Salesforce}. \\O mercado também exige \\que os desenvolvedores possam \\integrar organizações e\\redes de contatos em um ambiente \\de nuvem comercial.\end{tabular}  \\ 
\hline
Linux                                 & \begin{tabular}[c]{@{}l@{}}A maioria dos líderes de \\empresas de TI procuram \\por quem saiba operar \\em ambiente Linux. Ora, \\simplesmente porque \\existe muita gente utilizando \\sistemas operacionais \\e aplicativos desenvolvidos \\a partir do Linux, \\principalmente o que hoje é \\chamado de “Infraestrutura \\como Serviço”.\end{tabular}                                                                                                                                                                             \\ 

\hline
\end{tabular}
\end{table}
%---------------------
\begin{table}
\centering
\caption{Habilidades necessárias para se trabalhar com computação em nuvem}
\begin{tabular}{|l|l|} 
\hline
Puppet e Chef                   & \begin{tabular}[c]{@{}l@{}}As grandes corporações têm \\utilizado atualmente dois\\~softwares para o manuseio \\do Computação em nuvem. \\Um é o chamado “Puppet”,\\~usado na automação de TI; \\e o “Chef”, que serve para \\configurar índices e referências. \\Quem pretende trabalhar com \\a tecnologia também deve estar\\~ciente do que ambos fazem e são capazes.\end{tabular}                                                                                                                                                                                                                                                               \\ 
\hline
Criatividade para produzir APIs & \begin{tabular}[c]{@{}l@{}}De que adianta você conseguir \\gerenciar vasto conteúdo rapidamente \\se não há aplicações úteis e inteligentes? \\O mercado, portanto, não quer somente \\quem compreenda a tecnologia em nuvem, \\mas também quem possa criar APIs \\(Application Programming Interface), \\ou seja, maneiras de utilizar os dados \\de acordo com dispositivos, usuários, etc.\end{tabular}                                                                                                                                                                                                                                           \\ 
\hline
Especialidade em segurança      & \begin{tabular}[c]{@{}l@{}}Assim como toda companhia quer \\saber como manusear uma grande \\quantidade de dados em nuvem, ela \\também quer que isso aconteça num\\~ambiente seguro, sem que alguém \\indesejado tenha acesso ou gerencie \\essas informações. Então, apesar de \\estar na décima colocação, essa \\habilidade é uma das prioridades do setor. \\O Computação em nuvem ainda é \\algo recente, assim como os \\profissionais dedicados a essa\\~área. Como as empresas que \\não possuem estes profissionais \\especializados estão avançando \\mais devagar, a procura deve \\aumentar bastante nos próximos \\anos.\end{tabular}  \\
\hline
\end{tabular}
\end{table}