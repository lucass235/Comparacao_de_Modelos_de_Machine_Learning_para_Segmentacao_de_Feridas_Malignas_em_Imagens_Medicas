\section{DISCUSSÕES}

\subsection{Contextualização dos Resultados}

Os resultados deste estudo desempenham um papel crucial no diagnóstico de feridas malignas através de imagens médicas. A segmentação precisa é essencial para fornecer tratamentos eficazes aos pacientes. Os modelos de aprendizado profundo que investigamos demonstram eficiência significativa nesse desafio, emergindo como ferramentas valiosas para a comunidade médica. Ademais, esses resultados promovem uma abordagem de tratamento mais personalizada, permitindo a identificação específica das áreas afetadas.

\subsection{Implicações Práticas}

Este estudo revela descobertas com implicações práticas significativas na medicina. Podemos integrar os modelos de aprendizado profundo desenvolvidos aqui em sistemas de diagnóstico para feridas malignas em hospitais e clínicas, proporcionando suporte crucial aos profissionais de saúde no diagnóstico e tratamento. Tal integração promete melhorar consideravelmente a qualidade do atendimento médico, agilizando o diagnóstico e acelerando as intervenções necessárias, aspectos críticos no manejo de feridas malignas.


\subsection{Comparação com Estudos Anteriores}
Este estudo comparou seus resultados com pesquisas anteriores que empregaram modelos de aprendizado profundo na segmentação de feridas malignas. Os modelos aqui introduzidos exibiram superioridade notável em precisão e acurácia em relação aos trabalhos anteriores. Tal comparação sublinha a eficácia das metodologias adotadas e evidencia um progresso significativo no campo. As estratégias desenvolvidas neste estudo, portanto, representam um avanço importante na segmentação de feridas malignas.


\subsection{Limitações do Estudo}
Reconhecer as limitações deste estudo é crucial. Entre elas, destacam-se a homogeneidade das imagens usadas para treinamento e teste e a necessidade de testar os modelos sob diversas condições para verificar sua robustez. Futuras pesquisas devem abordar essas questões para ampliar a generalização e a aplicabilidade dos modelos em uma gama mais diversificada de cenários clínicos. Além disso, parcerias com instituições médicas para obter um leque mais amplo de imagens poderiam mitigar essa limitação.


\subsection{Sugestões para Pesquisas Futuras}

Este estudo revela insights valiosos e aponta para direções promissoras em pesquisas futuras. As possibilidades incluem a exploração de novas arquiteturas de modelos de aprendizado profundo, a incorporação de conjuntos de dados mais variados para treinamento e teste, e o exame de técnicas avançadas de aumento de dados. Estas recomendações têm o potencial de impulsionar avanços contínuos na precisão da segmentação de feridas, permitindo o desenvolvimento de modelos mais robustos e precisos, capazes de se ajustar a uma gama mais ampla de contextos clínicos.


\subsection{Conclusão}

Os resultados deste estudo revestem-se de crucial importância para o diagnóstico de feridas malignas através de imagens médicas. Os modelos de aprendizado profundo que apresentamos sobressaem como ferramentas eficazes na segmentação de tais feridas, oferecendo implicações práticas significativas para a comunidade médica. Embora tenhamos identificado limitações, as direções que sugerimos para pesquisas futuras delineiam um caminho promissor para o aprimoramento contínuo. Esta discussão resume as principais contribuições e desafios enfrentados durante o estudo, enfatizando a importância dos resultados e delineando os passos futuros para refinar ainda mais as abordagens propostas.

