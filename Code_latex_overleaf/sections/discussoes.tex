\section{DISCUSSÕES}

\subsection{Contextualização dos Resultados}

Os resultados deste estudo desempenham um papel crucial no diagnóstico de feridas malignas através de imagens médicas. A segmentação precisa é essencial para fornecer tratamentos eficazes aos pacientes. Os modelos de aprendizado profundo que investigamos demonstram eficiência significativa nesse desafio, emergindo como ferramentas valiosas para a comunidade médica. Ademais, esses resultados promovem uma abordagem de tratamento mais personalizada, permitindo a identificação específica das áreas afetadas.

\subsection{Implicações Práticas}

Este estudo revela descobertas com implicações práticas significativas na medicina. Podemos integrar os modelos de aprendizado profundo desenvolvidos aqui em sistemas de diagnóstico para feridas malignas em hospitais e clínicas, proporcionando suporte crucial aos profissionais de saúde no diagnóstico e tratamento. Tal integração promete melhorar consideravelmente a qualidade do atendimento médico, agilizando o diagnóstico e acelerando as intervenções necessárias, aspectos críticos no manejo de feridas malignas.


\subsection{Comparação com Estudos Anteriores}
Este estudo comparou seus resultados com pesquisas anteriores que empregaram modelos de aprendizado profundo na segmentação de feridas malignas. Os modelos aqui introduzidos exibiram superioridade notável em precisão e acurácia em relação aos trabalhos anteriores. Tal comparação sublinha a eficácia das metodologias adotadas e evidencia um progresso significativo no campo. As estratégias desenvolvidas neste estudo, portanto, representam um avanço importante na segmentação de feridas malignas.


\subsection{Limitações do Estudo}
É crucial reconhecer as limitações deste estudo. Notavelmente, as imagens utilizadas tanto para treinamento quanto para teste foram bastante homogêneas, por isso continua sendo necessário confirmar a força dos modelos testando-os sob diferentes condições. Responder a estas questões e expandir a generalização dos modelos para ambientes clínicos mais diversos deve ser o foco principal de pesquisas futuras. Além disso, a aquisição de uma gama mais ampla de imagens de instituições médicas através de parcerias poderia ajudar a resolver esta limitação.


\subsection{Sugestões para Pesquisas Futuras}

Direções promissoras para pesquisas futuras emergem deste estudo e revelam insights valiosos. Os avanços potenciais incluem o exame de técnicas avançadas de aumento de dados, a exploração de novas arquiteturas de modelos de aprendizagem profunda e a incorporação de conjuntos de dados mais variados para treinamento e teste. Através destas recomendações, a precisão da segmentação de feridas pode continuar a progredir, culminando no desenvolvimento de modelos robustos e precisos, capazes de se adaptarem a uma gama mais ampla de contextos clínicos.


\subsection{Conclusão}

\red{Os resultados deste estudo são de suma importância para o diagnóstico preciso de feridas malignas por meio de imagens médicas. Os modelos de aprendizado profundo apresentados destacam-se como ferramentas eficazes na segmentação dessas feridas, oferecendo implicações práticas significativas para a comunidade médica. A avaliação detalhada e comparativa dos modelos estudados revelou nuances importantes sobre suas performances. Essa análise minuciosa nos permitiu identificar o modelo \textit{\ac{FCN}} como a escolha mais sólida para a segmentação precisa de feridas malignas com foca na sua precisão do resultado.}

\red{Embora tenhamos reconhecido limitações em alguns modelos, a identificação do modelo \textit{\ac{FCN}} como o mais eficaz oferece uma orientação valiosa para os profissionais de saúde. Esse modelo especificamente demonstrou maior precisão, Recall e Coeficiente Dice, traduzindo-se em uma capacidade superior de identificar com precisão as áreas afetadas nas imagens médicas. Essa precisão aprimorada é crucial para diagnósticos mais precisos e intervenções clínicas mais eficazes.}

\red{Além disso, as direções apontadas para pesquisas futuras delineiam um caminho promissor para aprimoramentos contínuos. Propomos investigações mais aprofundadas na arquitetura do modelo \textit{\ac{FCN}} para explorar ainda mais suas capacidades de generalização e aprimoramento da precisão em diferentes contextos clínicos. Abrindo espaço também para expandir ainda mais a diversidade da base de dados utilizada. Incorporar uma variedade ainda maior de imagens de feridas malignas de diferentes fontes e condições clínicas poderia aprimorar significativamente a capacidade do modelo de lidar com uma gama mais ampla de cenários clínicos. Essa expansão da base de dados pode ajudar a fortalecer a robustez dos modelos, tornando-os mais capazes de generalizar e se adaptar a situações clínicas variadas.}

\red{Em suma, a identificação do modelo \textit{\ac{FCN}} como o mais indicado neste estudo por sua precisão oferece uma base sólida para os profissionais de saúde. Essa escolha pode resultar em diagnósticos mais precisos e tratamentos mais eficazes, contribuindo diretamente para a qualidade dos cuidados prestados aos pacientes com feridas malignas.}

