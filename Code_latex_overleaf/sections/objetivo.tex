\section{OBJETIVO}

\subsection{Objetivo Geral}

Este estudo busca avaliar a eficiência e aplicabilidade de modelos de aprendizado profundo, especificamente \ac{CNN}, na segmentação de feridas malignas em imagens médicas, com foco na precisão dos resultados.

\subsection{Objetivos Específicos}

\begin{itemize}
  \item Implementar modelos de aprendizado profundo, como \ac{FCN}\footnote{\url{https://paperswithcode.com/method/fcn}}, \ac{U-Net}\footnote{\url{https://paperswithcode.com/method/u-net}}, \ac{SegNet}\footnote{\url{https://paperswithcode.com/method/segnet}} e \ac{MobileNetV2}\footnote{\url{https://paperswithcode.com/method/mobilenetv2}}, para a segmentação de feridas malignas em imagens médicas.
  \item Otimizar o pré-processamento das imagens para melhorar a precisão e eficiência dos modelos.
  \item Avaliar o desempenho dos modelos com base em métricas como Loss, Precison, Recall e Coeficiente de Dice.
  \item Comparar a eficácia das diversas arquiteturas de aprendizado profundo na tarefa de segmentação.
  \item Fornecer diretrizes para futuras pesquisas em segmentação de imagens médicas com aprendizado profundo.
  \item Contribuir para a oncologia cutânea, desenvolvendo ferramentas automáticas e eficientes para a segmentação de feridas malignas.
\end{itemize}

% O restante deste artigo está organizado da seguinte maneira: Seção 2 apresenta a Fundamentação Teórica; Seção 3 Apresentar os Trabalhos Relacionados; Seção 4 Apresenta a Metodologia aplicada; Seção 5 Apresentar os Resultados obtidos; Seção 6 Explorar as Discussões da pesquisa; e Seção 7 Expor o Cronograma de Execução.



