\begin{center}
    \textbf{\\ABSTRACT \\ }
\end{center}

\textbf{Context}: The health sector increasingly demands precise segmentation of medical images, particularly in cutaneous oncology. In this context, the accurate and rapid identification of malignant wounds can result in more efficient treatments and favourable prognoses. Deep learning models, such as \acf{U-Net}, \acf{SegNet}, \acf{FCN} and \acf{MobileNetV2}, have been gaining ground in this scenario due to their capacity and application potential. \textbf{Objective}: This study aims to explore the efficiency and applicability of these deep learning models in the segmentation of malignant wounds in medical images, taking into account their results about the accuracy of the algorithms' results. \textbf{Method}: The methodology adopted for this research goes through a series of phases, from pre-processing the images to evaluating the performance of the deep learning models, including tests on different machine learning models. The performance of each model will be evaluated using established metrics such as loss, precision, recall and Dice coefficient, and the computational efficiency of each will be considered. \textbf{Results}: The results obtained in this study were promising; the models evaluated demonstrated high performance in the segmentation of malignant wounds and provided significant insights into the comparative performance between different deep learning architectures in medical applications. \textbf{Conclusion}: The findings of this study are expected to provide directions for future research in the field of medical image segmentation via deep learning. In addition, the research has the potential to bring notable benefits to medicine -- especially cutaneous oncology -- by providing automated and practical tools for segmenting malignant wounds, thus collaborating with diagnoses and treatment monitoring by health professionals.

\textbf{Keywords:} Cancer; Malignant Wounds; Wound Segmentation; Convolutional Neural Networks.