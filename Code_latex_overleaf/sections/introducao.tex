\section{INTRODUÇÃO}

Profissionais de saúde enfrentam um desafio considerável no tratamento de feridas crônicas, incluindo aquelas oriundas de processos oncológicos. Essas lesões exigem cuidados intensivos e, frequentemente, o objetivo central não é a cicatrização, mas a minimização do impacto dos sintomas na qualidade de vida dos pacientes~\cite{freitas2017intervenccoes, agra2017neoplastic}. No Brasil, estima-se que ocorram 704 mil novos casos de câncer anualmente para o biênio 2023-2025, com as feridas malignas representando um problema significativo~\cite{de2023estimativa}.

Tais feridas são prevalentes em cânceres de mama, cabeça e pescoço, surgindo da infiltração de células oncológicas na pele e resultando em lesões exofíticas. Estas não só desfiguram progressivamente o paciente, mas também são friáveis, dolorosas, exsudativas e malcheirosas, complicando ainda mais o quadro clínico~\cite{de2015manejo}. A avaliação tradicional manual dessas feridas é um processo subjetivo, demorado e impreciso. Contudo, os avanços tecnológicos em dispositivos móveis e de armazenamento de dados oferecem métodos alternativos para uma avaliação mais precisa, como a segmentação automática e a medição do tamanho da ferida assistidas por computador~\cite{scebba2022detect}.

A \ac{IA}\footnote{\url{https://aws.amazon.com/pt/what-is/artificial-intelligence/}}, e mais especificamente as \ac{CNN}\footnote{\url{https://medium.com/neuronio-br/entendendo-redes-convolucionais-cnns-d10359f21184}}, emergem como uma solução promissora para superar as limitações da avaliação manual~\cite{litjens2017, lundervold2019, esteva2019}. As \acp{CNN}, uma subcategoria das \ac{ANN}, têm alcançado sucesso em tarefas de visão computacional, como classificação de imagens, detecção de objetos e segmentação semântica~\cite{sun2023convolution}. Elas podem aprender a diferenciar a ferida da pele saudável adjacente e identificar características cruciais, proporcionando avaliações mais objetivas e consistentes.

Este estudo objetiva investigar a aplicação e eficácia das \acp{CNN} na segmentação e diagnóstico de feridas malignas, visando otimizar o tratamento por meio de avaliações mais precisas e objetivas. Almejamos desenvolver ferramentas assistivas integráveis à prática clínica, aprimorando o atendimento e potencialmente acelerando a cicatrização. A pesquisa justifica-se pela necessidade de superar os desafios na avaliação de feridas malignas e explorar o potencial das tecnologias de \ac{IA} no cuidado à saúde. Espera-se que a incorporação da \ac{IA} avançada nas práticas de avaliação de feridas não apenas melhore a precisão, mas também reduza o tempo de resposta, permitindo que os profissionais de saúde dediquem-se mais ao desenvolvimento de estratégias de tratamento eficazes.




