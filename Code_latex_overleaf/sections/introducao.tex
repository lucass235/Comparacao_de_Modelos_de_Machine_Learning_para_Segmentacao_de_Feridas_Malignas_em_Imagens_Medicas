\section{INTRODUÇÃO}
No Brasil, estima-se que anualmente, entre 2023 e 2025, ocorrerão aproximadamente 704.000 novos casos de câncer. As feridas malignas, decorrentes destes cânceres, representam um desafio significativo no tratamento médico~\cite{de2023estimativa}. Profissionais de saúde lidam com a complexidade dessas feridas crônicas, frequentemente oriundas de processos oncológicos, que exigem cuidados intensivos focados não apenas na cicatrização, mas também na melhoria da qualidade de vida dos pacientes~\cite{freitas2017intervenccoes, agra2017neoplastic}.

As \ac{CNNs}, subcategoria das \ac{ANN}, têm se mostrado eficazes em diversas aplicações de visão computacional, como classificação e segmentação de imagens~\cite{sun2023convolution}. Essas redes conseguem distinguir lesões de pele saudável adjacente, identificando características essenciais para avaliações mais objetivas e consistentes~\cite{litjens2017, lundervold2019, esteva2019}.

\subsection{Apresentação do Problema}
Feridas malignas, frequentemente associadas a cânceres de mama, cabeça e pescoço, surgem da infiltração agressiva de células cancerígenas na pele, levando ao desenvolvimento de lesões exofíticas que alteram progressivamente a aparência do paciente. Estas lesões caracterizam-se por sua natureza friável, onde pequenos traumas podem causar sangramentos significativos. Além disso, são notoriamente dolorosas, afetando diretamente o bem-estar e a qualidade de vida do indivíduo afetado.

As feridas malignas são também conhecidas por serem exsudativas, produzindo um exsudato que pode ser desafiador de gerenciar. Este exsudato, muitas vezes abundante, requer cuidados constantes e especializados para prevenir infecções e manter a integridade da pele ao redor. Outro aspecto desafiador é o odor desagradável que muitas vezes acompanha estas feridas, resultado da necrose tecidual e da colonização bacteriana, o que pode ter um impacto psicológico profundo tanto nos pacientes quanto em seus cuidadores.

No contexto clínico, a avaliação manual tradicional dessas feridas apresenta várias limitações. O processo é intrinsecamente subjetivo, dependendo amplamente da experiência e da percepção individual do profissional de saúde. Esta subjetividade pode levar a variações consideráveis na avaliação, dificultando o estabelecimento de um plano de tratamento consistente e eficaz. Além disso, a avaliação manual é frequentemente morosa, exigindo tempo considerável tanto para a inspeção visual quanto para o toque cuidadoso, visando evitar dor adicional ou sangramento no paciente.

O problema da segmentação de imagens médicas, como o de feridas malignas, é um desafio importante na área da saúde. A segmentação refere-se ao processo de identificar e delimitar as áreas de interesse em uma imagem, neste caso, as áreas afetadas por feridas malignas. A segmentação precisa é fundamental para o diagnóstico e tratamento adequados, pois permite que os médicos e profissionais de saúde avaliem com precisão o tamanho, a forma e a localização das feridas. No entanto, a segmentação neste contexto é um problema complexo devido à variação na aparência e forma das feridas, bem como à presença de ruído e artefatos nas imagens. Além disso, a segmentação manual é um processo demorado e sujeito a erros, o que pode levar a diagnósticos imprecisos e tratamentos inadequados. Por isso, a utilização de algoritmos de aprendizado profundo, como os explorados neste estudo, pode ser uma solução promissora para a segmentação de feridas malignas em imagens médicas.

A precisão da avaliação manual também é uma preocupação, já que imprecisões podem levar a diagnósticos errôneos ou a atrasos no tratamento adequado. Devido à complexidade e à natureza dinâmica dessas feridas malignas, é imperativo buscar métodos que possam complementar ou melhorar a precisão da avaliação clínica~\cite{de2023estimativa}. Portanto, torna-se essencial explorar novas tecnologias e abordagens, como as Redes Neurais Convolucionais, para melhorar a precisão e a eficiência na avaliação destas feridas desafiadoras.

Assim, este trabalho tem como objetivo explorar o uso de algoritmos de aprendizado profundo para a segmentação automática de feridas malignas em imagens médicas. Acreditamos que a utilização desses algoritmos pode ajudar a melhorar a eficiência e precisão do diagnóstico e tratamento de feridas malignas, além de reduzir o tempo e o esforço necessários para a segmentação manual. Além disso, a segmentação automática pode ser a primeira etapa de um sistema de análise automática de feridas, permitindo que as etapas subsequentes se concentrem apenas na área da ferida, sem serem afetadas por artefatos que não fazem parte da ferida.

\subsection{Questões de Pesquisa}
 A investigação explora se o \ac{CNN} pode melhorar significativamente a precisão na avaliação de feridas malignas e se a sua integração na prática clínica pode acelerar o processo de cicatrização e otimizar o tratamento. %\section{Metodologia (Breve descrição geral)} 
 O estudo adaptará uma abordagem quantitativa, utilizando técnicas avançadas de \ac{IA} para analisar imagens de feridas malignas. A eficácia da \ac{CNN} será avaliada através da comparação dos resultados com as avaliações manuais tradicionais.
\subsection{Objetivo Geral}
Este estudo busca avaliar a eficiência e aplicabilidade de modelos de aprendizado profundo, especificamente \ac{CNN}, na segmentação de feridas malignas em imagens médicas, com foco na precisão dos resultados.

\subsection{Objetivos Específicos}
Este estudo tem como objetivo investigar a aplicação e eficácia das \ac{CNN} na segmentação de feridas malignas. O objetivo é otimizar o tratamento através de avaliações mais precisas e objectivas, desenvolver ferramentas de apoio que possam ser integradas na prática clínica, melhorar os cuidados e acelerar a cicatrização.

\begin{itemize}
  \item Implementar modelos de aprendizado profundo, como \ac{FCN}\footnote{\url{https://paperswithcode.com/method/fcn}}, \ac{U-Net}\footnote{\url{https://paperswithcode.com/method/u-net}}, \ac{SegNet}\footnote{\url{https://paperswithcode.com/method/segnet}} e \ac{MobileNetV2}\footnote{\url{https://paperswithcode.com/method/mobilenetv2}}, para a segmentação de feridas malignas em imagens médicas.
  \item Otimizar o pré-processamento das imagens para melhorar a precisão e eficiência dos modelos.
  \item Avaliar o desempenho dos modelos com base em métricas como Loss, Precison, Recall e Coeficiente de Dice.
  \item Comparar a eficácia das diversas arquiteturas de aprendizado profundo na tarefa de segmentação.
  \item Fornecer diretrizes para futuras pesquisas em segmentação de imagens médicas com aprendizado profundo.
  \item Contribuir para a oncologia cutânea, desenvolvendo ferramentas automáticas e eficientes para a segmentação de feridas malignas.
\end{itemize}

\subsection{Estrutura do TCC}

A estrutura deste trabalho segue uma progressão lógica e abrangente. Iniciando com esta introdução, o TCC segue algumas etapas:

Revisão da literatura, a qual oferece uma análise detalhada dos estudos relacionados à segmentação de feridas malignas em imagens médicas. Este capítulo aborda os principais desafios encontrados na área e explora as técnicas mais utilizadas, oferecendo um panorama abrangente do estado da arte.

A metodologia é então apresentada, detalhando o processo adotado para a realização deste estudo. Este capítulo engloba desde a coleta até o pré-processamento dos dados, a seleção dos modelos de aprendizado profundo e os parâmetros de treinamento, além de descrever as métricas de avaliação utilizadas para analisar o desempenho dos modelos.

Os resultados obtidos pelos modelos de aprendizado profundo avaliados neste estudo são minuciosamente apresentados, incluindo métricas essenciais como Precisão, Recall, Coeficiente Dice e a perda associada a cada modelo. Esses resultados são então discutidos em profundidade, considerando suas implicações para a segmentação de feridas malignas em imagens médicas, bem como destacando as limitações e possíveis direções para pesquisas futuras.

Por fim, o trabalho conclui com um capítulo de conclusões, ressaltando os insights e descobertas mais significativos e fornecendo recomendações valiosas para investigações posteriores na área. Esta estrutura oferece uma abordagem abrangente e analítica, permitindo uma compreensão profunda e crítica do tema explorado ao longo deste TCC.