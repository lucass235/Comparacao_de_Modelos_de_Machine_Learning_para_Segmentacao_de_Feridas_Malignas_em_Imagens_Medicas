\section{INTRODUÇÃO}
No Brasil, estima-se que anualmente, entre 2023 e 2025, ocorrerão aproximadamente 704.000 novos casos de câncer. As feridas malignas, decorrentes destes cânceres, representam um desafio significativo no tratamento médico~\cite{de2023estimativa}. Profissionais de saúde lidam com a complexidade dessas feridas crônicas, frequentemente oriundas de processos oncológicos, que exigem cuidados intensivos focados não apenas na cicatrização, mas também na melhoria da qualidade de vida dos pacientes~\cite{freitas2017intervenccoes, agra2017neoplastic}.

\red{Com esses desafios enfrentados pelos profissionais de saúde é buscado alternativas eficazes para auxiliar esses profissionais nestas questões.} As \acp{CNN}, subcategoria das \ac{ANN}, têm se mostrado eficazes em diversas aplicações de visão computacional, como classificação e segmentação de imagens~\cite{sun2023convolution}. Essas redes conseguem distinguir lesões de pele saudável adjacente, identificando características essenciais para avaliações mais objetivas e consistentes~\cite{litjens2017, lundervold2019, esteva2019}.

\subsection{Apresentação do Problema}

\red{As feridas malignas em várias áreas do corpo são frequentemente associadas a condições dolorosas e impactantes. A infiltração agressiva de células cancerígenas na pele resulta em lesões que alteram a aparência do paciente. Essas lesões são especialmente delicadas, podendo causar sangramentos excessivos com pequenos traumas. Além disso, provocam desconforto considerável, afetando diretamente a qualidade de vida e o bem-estar dos indivíduos afetados.} \red{\cite{kathleen2017topical}} 

As feridas malignos também são conhecidas por serem exsudativas, produzindo exsudato de difícil controle. Este exsudado é frequentemente abundante e requer cuidados especializados contínuos para prevenir infeções e manter a integridade da pele circundante. Outro aspecto desafiador é o mau odor associado a essas feridas, resultado da necrose tecidual e da colonização bacteriana. Isso pode ter um impacto psicológico significativo tanto nos pacientes quanto em seus cuidadores. \red{\cite{firmino2021topical}}

No contexto clínico estas autoavaliações tradicionais do trauma têm várias limitações. Este processo é pessoal. Depende principalmente da experiência e percepção de cada pessoal médico. Esta subjetividade pode levar a uma variabilidade significante na avaliação, dificultando a criação de um plano de tratamento consistente e eficaz. Além disso, a avaliação manual costuma ser demorada, demandando um tempo significante tanto para inspeção visual quanto para toque suave para evitar mais dor ou sangramento ao paciente. \red{\cite{ramasubbu2017systemic}}

A dificuldade de segmentar imagens médicas, como as de feridas dolorosos, tornou-se um grande obstáculo na área da saúde. Segmentação é o processo de categorizar e separar partes de uma imagem relacionadas a determinadas áreas de interesse. Neste caso, a região de interesse é a parte do quadro afetada pela ferida maligna. A segmentação eficaz é fundamental para um diagnóstico e tratamento precisos, pois permite que médicos e outros profissionais de saúde determinam com acurácia o tamanho a forma e a localização das feridas. Porém, a segmentação neste contexto representa um problema desafiador devido à variedade de manifestações e formas de trauma, bem como à presença de ruídos e artefatos nas imagens. Além disso, a segmentação manual é demorada e sujeita a erros Esse processo pode levar a diagnósticos errados e tratamento inadequado. Portanto, a implementação de algoritmos de aprendizagem profunda como o descrito neste estudo pode ser considerada uma abordagem promissora para segmentação de cicatrizes malignos em imagens médicas. \red{\cite{wang2018interactive}}

A acurácia da autoavaliação também é uma preocupação. Isso ocorre porque os desvios podem levar a enganos de diagnose ou atrasos no tratamento correto. Isto se deve à complexidade e à dinâmica dessas feridas mortais. Consequentemente, é fundamental encontrar técnicos que possam aprimorar ou melhorar a acurácia das avaliações clínicas. Consequentemente, é imperativo explorar novas tecnologias e métodos, como redes neurais artificiais. Para melhorar a precisão e a eficiência da avaliação dessas feridas complexos.

Consequentemente, este trabalho tem como objetivo explorar o uso de algoritmos de aprendizagem profunda para segmentação automática de feridas malignas em imagens médicas. \red{Foi verificado} que o uso desses algoritmos pode ajudar a melhorar a eficiência e a acurácia do diagnóstico e tratamento de feridas malignos, bem como reduzir o tempo e o esforço necessários para a segmentação manual. Além disso, a segmentação automática pode ser o primeiro passo em um sistema automatizado de análise de feridas, permitindo que as etapas subsequentes se concentrem apenas na área da ferida sem serem afetadas por artefatos que não fazem parte da ferida.

 \subsection{Questões de Pesquisa}
Neste estudo, é investigado questões científicas detalhadas sobre a aplicação de \ac{CNN} na segmentação de feridas malignas em imagens médicas:

\begin{enumerate}
  \item \textbf{QP1}: \textit{“Até que ponto as Redes Neurais Convolucionais avançadas melhoram a precisão na segmentação de feridas malignas em comparação com técnicas tradicionais de análise de imagens médicas, e quais fatores contribuem para sua eficácia em diferentes contextos clínicos e tipos de feridas?”}
\end{enumerate}

A abordagem será quantitativa, empregando técnicas avançadas de \ac{IA} para analisar e interpretar imagens médicas. Além disso, a eficácia das \ac{CNN}s será mensurada e validada por meio da comparação com métodos de avaliação manual convencionais realizados por profissionais da saúde. Esses questão de pesquisa pode ser melhor respondido ao longo do trabalho, podemos ver uma explicação mais detalhadas no final da subtópico \ref{sec:perguntas} da seção de Resultados.


\subsection{Objetivo Geral}
Este estudo busca avaliar a eficiência e aplicabilidade de modelos de aprendizado profundo, especificamente \ac{CNN}, na segmentação de feridas malignas em imagens médicas, com foco na precisão dos resultados.

\subsection{Objetivos Específicos}
Este estudo tem como objetivo investigar a aplicação e eficácia das \acp{CNN} na segmentação de feridas malignas. O objetivo é otimizar o tratamento através de avaliações mais precisas e objectivas, desenvolver ferramentas de apoio que possam ser integradas na prática clínica, melhorar os cuidados e acelerar a cicatrização.

\begin{itemize}
  \item Implementar modelos de aprendizado profundo, como \ac{FCN}\footnote{\url{https://paperswithcode.com/method/fcn}}, \ac{U-Net}\footnote{\url{https://paperswithcode.com/method/u-net}}, \ac{SegNet}\footnote{\url{https://paperswithcode.com/method/segnet}} e \ac{MobileNetV2}\footnote{\url{https://paperswithcode.com/method/mobilenetv2}}, para a segmentação de feridas malignas em imagens médicas.
  \item Otimizar o pré-processamento das imagens para melhorar a precisão e eficiência dos modelos.
  \item Avaliar o desempenho dos modelos com base em métricas como Loss, Precison, Recall e Coeficiente de Dice.
  \item Comparar a eficácia das diversas arquiteturas de aprendizado profundo na tarefa de segmentação.
  \item Fornecer diretrizes para futuras pesquisas em segmentação de imagens médicas com aprendizado profundo.
  \item Contribuir para a oncologia cutânea, desenvolvendo ferramentas automáticas e eficientes para a segmentação de feridas malignas.
\end{itemize}

\subsection{Estrutura do TCC}

A estrutura deste trabalho segue uma progressão lógica e abrangente. Iniciando com esta introdução, o TCC segue algumas etapas:

Revisão da literatura, a qual oferece uma análise detalhada dos estudos relacionados à segmentação de feridas malignas em imagens médicas. Este capítulo aborda os principais desafios encontrados na área e explora as técnicas mais utilizadas, oferecendo um panorama abrangente do estado da arte.

A metodologia é então apresentada, detalhando o processo adotado para a realização deste estudo. Este capítulo engloba desde a coleta até o pré-processamento dos dados, a seleção dos modelos de aprendizado profundo e os parâmetros de treinamento, além de descrever as métricas de avaliação utilizadas para analisar o desempenho dos modelos.

Os resultados obtidos pelos modelos de aprendizado profundo avaliados neste estudo são minuciosamente apresentados, incluindo métricas essenciais como Precisão, Recall, Coeficiente Dice e a perda associada a cada modelo. Esses resultados são então discutidos em profundidade, considerando suas implicações para a segmentação de feridas malignas em imagens médicas, bem como destacando as limitações e possíveis direções para pesquisas futuras.

Por fim, o trabalho conclui com um capítulo de conclusões, ressaltando os insights e descobertas mais significativos e fornecendo recomendações valiosas para investigações posteriores na área. Esta estrutura oferece uma abordagem abrangente e analítica, permitindo uma compreensão profunda e crítica do tema explorado ao longo deste \ac{TCC}.