\section{TRABALHOS RELACIONADOS}

Esta seção apresenta estudos relacionados ao uso de \ac{CNN} para a segmentação de imagens médicas, oferecendo um panorama das pesquisas recentes e estabelecendo um comparativo com o presente trabalho. É também demostrado as bases de dados exploradas em cada estudo. Podemos ver essa análise na tabela ~\ref{tab:analiseTrabalhosRelacionados}.

\begin{table}[htbp]
\tiny
\centering
\caption{Análise Comparativa dos Trabalhos Relacionados}
\label{tab:analiseTrabalhosRelacionados}
\begin{tblr}{
  cells = {c},
  row{9} = {Silver},
  cell{1}{1} = {r=2}{},
  cell{1}{2} = {c=4}{},
  cell{1}{6} = {c=4}{},
  vlines,
  hline{1,3-10} = {-}{},
  hline{2} = {2-9}{},
}
\textbf{Autores } & \textbf{Datasets} &&&& \textbf{Models} &&&\\

& {\textbf{Wound}\\\textbf{Segmentation}} & \textbf{WSNET} & {\textbf{Venous leg ulcers}\\\textbf{and arterial leg ulcers}} & {\textbf{Foot Wounds}\\\textbf{and Ulcers}} & \textbf{U-Net}  & \textbf{SegNet} & \textbf{MobileNetV2} & \textbf{FCN} \\

\cite{li2020fully}          & X &   &   &   & X &   & X &    \\
\cite{silva2021avaliacao}   & X &   &   &   &   &   & X & X  \\
\cite{liu2021computational} & X &   &   & X &   & X &   & X  \\
 \cite{akay2021deep}        &   &   & X &   &   &   & X &     \\
 \cite{mahbod2022automatic} & X &   &   &   & X &   &   &     \\
 \cite{prakash2023end}      &   & X &   &   & X &   &   & X    \\
\textbf{Nosso trabalho}     & \textbf{X} & \textbf{X} & \textbf{X} & \textbf{X} & \textbf{X} & \textbf{X} & \textbf{X} & \textbf{X}
\end{tblr}
\end{table}

Em 2020, Li et al, propuseram uma abordagem inovadora para a segmentação automática de feridas em imagens naturais, utilizando uma rede neural profunda com base no modelo \ac{MobileNetV2}. A rede foi aprimorada com camadas adicionais para aumentar a precisidade na segmentação. Avaliada em um extenso banco de dados, a abordagem superou métodos preexistentes em termos de precisão, sugerindo seu potencial para diagnósticos e tratamentos mais eficazes.~\cite{li2020fully}

Silva e colaboradores, em 2021, avaliaram dois modelos de aprendizado profundo, \ac{U-Net} e DeeplabV3, para segmentação de feridas malignas cutâneas. Utilizando um conjunto de dados do AZH Wound and Vascular Center, o modelo \ac{U-Net} demonstrou superioridade, enquanto o DeeplabV3 mostrou-se competitivo. O estudo também explorou a relação entre acurácia e eficiência computacional, apontando para futuras pesquisas focadas na compactação de redes neurais para dispositivos com restrições de hardware.~\cite{silva2021avaliacao}

No mesmo ano, Liu et al, realizaram uma revisão sistemática sobre metodologias computacionais aplicadas à medição e diagnóstico de feridas, destacando o papel das tecnologias de IA. A revisão abrangeu mais de 250 artigos, dos quais 115 foram selecionados por sua relevância. O estudo enfatizou a importância das tecnologias emergentes para a avaliação de feridas, concluindo que elas podem aumentar a precisão e eficiência no tratamento.~\cite{liu2021computational}

Em 2021 Akay, M., Du, Y, apresentou uma nova rede de aprendizado profundo para a caracterização da pele de pacientes com Esclerose Sistêmica (SSc), uma doença autoimune rara. A rede proposta é baseada no modelo \ac{MobileNetV2} e é capaz de realizar a classificação de imagens de pele com alta precisão, o que pode ajudar no diagnóstico precoce da doença. O artigo discute os desafios enfrentados na aplicação de redes neurais profundas em aplicações médicas, como a falta de dados de treinamento e a necessidade de computação de alto desempenho. A rede proposta é projetada para trabalhar com poucas imagens de treinamento e fornecer classificações mais precisas.~\cite{akay2021deep}

No ano de 2022 o Mahbod et al, propôs um método de segmentação automática de úlceras nos pés usando uma abordagem de conjunto de redes neurais convolucionais (CNNs). O método proposto utiliza dois modelos de CNN, o LinkNet e o \ac{U-Net}, para melhorar a precisão da segmentação de úlceras nos pés. O LinkNet e o \ac{U-Net} são modelos de CNN baseados em codificador-decodificador que têm mostrado excelente desempenho em tarefas de análise de imagens médicas, incluindo segmentação de imagens médicas.~\cite{mahbod2022automatic}

Recentemente, em 2023, Prakash et al, introduziram um framework de aprendizado profundo para a segmentação automática de lesões em imagens de ressonância magnética. O framework GA-UNet mostrou-se eficaz na segmentação e quantificação de áreas afetadas por lesões cerebrais traumáticas, ressaltando a importância da segmentação precisa para intervenções terapêuticas.~\cite{prakash2023end}


\subsection{Limitação dos Trabalhos Relacionados}
A literatura existente sobre a utilização de CNNs para segmentação de imagens médicas revela avanços significativos, no entanto, persistem lacunas importantes. Muitos estudos focam no desenvolvimento de modelos sem abordar integralmente as complexidades inerentes às imagens médicas, como variações na forma, tamanho e textura das lesões. Essa limitação pode resultar em imprecisões na segmentação, afetando diretamente a qualidade do diagnóstico e do tratamento. Além disso, a heterogeneidade dos dados clínicos, que inclui variáveis como diferentes tipos de lesões, a qualidade das imagens e a diversidade dos pacientes, não é suficientemente considerada. Essa negligência pode comprometer a capacidade de generalização dos modelos e sua eficácia em ambientes clínicos reais, que apresentam uma gama mais ampla de variáveis.

Outro ponto crítico é a falta de comparação abrangente com outros métodos de segmentação de imagens. A ênfase predominante em modelos de aprendizado profundo sem análises comparativas limita a compreensão da eficácia relativa desses métodos frente a abordagens tradicionais, como as baseadas em regras ou em detecção de bordas. Por fim, a validação dos modelos em condições clínicas reais é frequentemente insuficiente, levando a uma compreensão limitada da aplicabilidade prática dessas soluções. Muitos estudos se concentram em ambientes controlados de laboratório, sem testar a robustez e a adaptabilidade dos modelos em situações clínicas mais variadas e imprevisíveis.


\subsection{Diferencial do Nosso Trabalho }
Em nosso estudo, propomos uma abordagem inovadora e abrangente para a segmentação de feridas malignas em imagens médicas, combinando a eficiência de múltiplos modelos de aprendizado profundo com técnicas avançadas de pré-processamento de imagens. Investigamos especificamente a aplicabilidade de quatro modelos de \ac{CNN}: \ac{FCN}, \ac{U-Net}, \ac{SegNet} e \ac{MobileNetV2}. - para abordar as limitações identificadas em trabalhos anteriores.

Diferenciando-nos, empregamos um conjunto de dados clínicos heterogêneo que abrange uma ampla gama de tipos de lesões, idades de pacientes e níveis de gravidade. Essa diversidade garante uma avaliação mais representativa dos modelos em cenários clínicos variados. Ademais, aplicamos técnicas de pré-processamento como normalização, aumento de dados e segmentação manual para melhorar a qualidade e consistência dos dados, aumentando assim a precisão e a generalização dos modelos. 

Utilizamos uma metodologia de validação cruzada estratificada, garantindo uma avaliação equitativa e abrangente dos modelos em diferentes conjuntos de dados. Essa abordagem assegura que os modelos sejam testados em condições diversas, refletindo melhor a realidade clínica.

Os resultados evidenciaram que nossa metodologia não só alcança uma alta precisão na segmentação de feridas malignas mas também demonstra uma notável capacidade de generalização em diversos cenários clínicos. Isso sugere que nossa abordagem tem um potencial significativo para aplicação em contextos médicos variados, contribuindo para diagnósticos e tratamentos mais precisos e eficazes.
