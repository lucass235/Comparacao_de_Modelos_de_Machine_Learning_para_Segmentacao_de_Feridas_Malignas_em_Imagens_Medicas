\section{TRABALHOS RELACIONADOS}

Esta seção apresenta estudos relacionados ao uso de \ac{CNN} para a segmentação de imagens médicas, oferecendo um panorama das pesquisas recentes e estabelecendo um comparativo com o presente trabalho. É também demostrado as bases de dados exploradas em cada estudo. Podemos ver essa análise na tabela ~\ref{tab:analiseTrabalhosRelacionados}.

\begin{table}[htbp]
\tiny
\centering
\caption{Análise Comparativa dos Trabalhos Relacionados}
\label{tab:analiseTrabalhosRelacionados}
\begin{tblr}{
  cells = {c},
  row{9} = {Silver},
  cell{1}{1} = {r=2}{},
  cell{1}{2} = {c=4}{},
  cell{1}{6} = {c=4}{},
  vlines,
  hline{1,3-10} = {-}{},
  hline{2} = {2-9}{},
}
\textbf{Autores } & \textbf{Datasets} &&&& \textbf{Models} &&&\\

& {\textbf{Wound}\\\textbf{Segmentation}} & \textbf{WSNET} & {\textbf{Venous leg ulcers}\\\textbf{and arterial leg ulcers}} & {\textbf{Foot Wounds}\\\textbf{and Ulcers}} & \textbf{U-Net}  & \textbf{SegNet} & \textbf{MobileNetV2} & \textbf{FCN} \\

\cite{li2020fully}          & X &   &   &   & X &   & X &    \\
\cite{silva2021avaliacao}   & X &   &   &   &   &   & X & X  \\
\cite{liu2021computational} & X &   &   & X &   & X &   & X  \\
 \cite{akay2021deep}        &   &   & X &   &   &   & X &     \\
 \cite{mahbod2022automatic} & X &   &   &   & X &   &   &     \\
 \cite{prakash2023end}      &   & X &   &   & X &   &   & X    \\
\textbf{Nosso trabalho}     & \textbf{X} & \textbf{X} & \textbf{X} & \textbf{X} & \textbf{X} & \textbf{X} & \textbf{X} & \textbf{X}
\end{tblr}
\par\medskip\textbf{Fonte:} Autor
\end{table}

Em 2020, Li et al, propuseram uma abordagem inovadora para a segmentação automática de feridas em imagens naturais, utilizando uma rede neural profunda com base no modelo \ac{MobileNetV2}. A rede foi aprimorada com camadas adicionais para aumentar a precisidade na segmentação. Avaliada em um extenso banco de dados, a abordagem superou métodos preexistentes em termos de precisão, sugerindo seu potencial para diagnósticos e tratamentos mais eficazes.~\cite{li2020fully}

Silva e colaboradores, em 2021, avaliaram dois modelos de aprendizado profundo, \ac{U-Net} e DeeplabV3, para segmentação de feridas malignas cutâneas. Utilizando um conjunto de dados do AZH Wound and Vascular Center, o modelo \ac{U-Net} demonstrou superioridade, enquanto o DeeplabV3 mostrou-se competitivo. O estudo também explorou a relação entre acurácia e eficiência computacional, apontando para futuras pesquisas focadas na compactação de redes neurais para dispositivos com restrições de hardware.~\cite{silva2021avaliacao}

No mesmo ano, Liu et al, realizaram uma revisão sistemática sobre metodologias computacionais aplicadas à medição e diagnóstico de feridas, destacando o papel das tecnologias de IA. A revisão abrangeu mais de 250 artigos, dos quais 115 foram selecionados por sua relevância. O estudo enfatizou a importância das tecnologias emergentes para a avaliação de feridas, concluindo que elas podem aumentar a precisão e eficiência no tratamento.~\cite{liu2021computational}

Em 2021 Akay, M., Du, Y, apresentou uma nova rede de aprendizado profundo para a caracterização da pele de pacientes com Esclerose Sistêmica (SSc), uma doença autoimune rara. A rede proposta é baseada no modelo \ac{MobileNetV2} e é capaz de realizar a classificação de imagens de pele com alta precisão, o que pode ajudar no diagnóstico precoce da doença. O artigo discute os desafios enfrentados na aplicação de redes neurais profundas em aplicações médicas, como a falta de dados de treinamento e a necessidade de computação de alto desempenho. A rede proposta é projetada para trabalhar com poucas imagens de treinamento e fornecer classificações mais precisas.~\cite{akay2021deep}

No ano de 2022 o Mahbod et al, propôs um método de segmentação automática de úlceras nos pés usando uma abordagem de conjunto de redes neurais convolucionais (CNNs). O método proposto utiliza dois modelos de CNN, o LinkNet e o \ac{U-Net}, para melhorar a precisão da segmentação de úlceras nos pés. O LinkNet e o \ac{U-Net} são modelos de CNN baseados em codificador-decodificador que têm mostrado excelente desempenho em tarefas de análise de imagens médicas, incluindo segmentação de imagens médicas.~\cite{mahbod2022automatic}

Recentemente, em 2023, Prakash et al, introduziram um framework de aprendizado profundo para a segmentação automática de lesões em imagens de ressonância magnética. O framework GA-UNet mostrou-se eficaz na segmentação e quantificação de áreas afetadas por lesões cerebrais traumáticas, ressaltando a importância da segmentação precisa para intervenções terapêuticas.~\cite{prakash2023end}


\subsection{Limitação de trabalhos relacionados}

    A literatura existente sobre o uso de \ac{CNNs} para segmentação de imagens médicas mostra progresso significante, mas ainda existem lacunas significantes. Muitos estudos se concentraram no desenvolvimento de modelos sem considerar as complexidades inerentes às imagens médicas, como diferenças no formato tamanho e textura das feridas. Essa limitação pode levar a imprecisões de segmentação, o que pode impactar diretamente na qualidade da diagnose e do tratamento. Além disso, a heterogeneidade dos dados clínicos, que inclui variáveis como diferentes tipos de lesões, qualidade de imagem e diversidade de pacientes, não é suficientemente levada em consideração. Esta negligência pode afetar a generalização dos modelos e a sua eficácia em ambientes clínicos reais que possuem uma gama mais ampla de variáveis.

    Outra crítica é a falta de uma comparação abrangente com outros métodos de segmentação de imagens. A ênfase dominante em modelos de aprendizagem profunda sem benchmarking limita a compreensão da eficácia relativa desses métodos em comparação com abordagens tradicionais, como a baseada em regras ou a detecção de bordas. Finalmente, a validação de modelos em ambientes clínicos do mundo real é muitas vezes insuficiente. Como resultado, há uma compreensão limitada da aplicação prática destas soluções. Muitos estudos concentram-se em ambientes laboratoriais controlados sem testar a robustez e adaptabilidade dos modelos em situações clínicas mais diversas e imprevisíveis.

\subsection{Diferenciação do nosso trabalho} 

    Em nosso estudo Declaramos uma abordagem inovadora e abrangente para segmentação de feridas malignas em imagens médicas. Ele combina o desempenho de vários modelos de aprendizagem profunda com técnicos avançadas de pré-processamento de imagens. Examinamos especificamente a aplicabilidade de quatro modelos \ac{CNN}: \ac{FCN}, \ac{U-Net}, \ac{SegNet} e \ac{MobileNetV2}, onde objetivamos sanar as limitações identificadas em trabalhos anteriores.

    Para nos diferenciarmos, usamos um conjunto de dados clínicos heterogêneos que inclui ampla variedade de tipos de lesões, idades e gravitação dos pacientes. Essa diversidade permite que o modelo seja avaliado em mais diferentes situações clínicas. Além disso, utilizamos técnicos de pré-processamento como normalização, aumento de dados e segmentação manual para melhorar a qualidade e consistência dos dados aumentando assim a precisão e generalização dos modelos.

    Usamos uma metodologia de validação cruzada em camadas para avaliar de forma justa e abrangente nossos modelos em diversos conjuntos de dados. Essa abordagem garante que o modelo seja testado em vários ambientes. e refletir melhor a realidade clínica.

    Os resultados mostram que nossa metodologia não só alcança alta precisão na segmentação de feridas malignos, mas também apresenta notável capacitância de generalização em diferentes cenários clínicos. Isto sugere que a nossa abordagem tem um potencial significante para aplicação em vários contextos médicos, contribuindo para diagnósticos e tratamentos mais precisos e eficazes.
