%
% ********** Página de Rosto
%

% titlepage gera páginas sem numeração
\begin{titlepage}

\begin{center}

\small

% O comando @{} no ambiente tabular x é para criar um novo delimitador
% entre colunas que não a barra vertical | que é normalmente utilizada.
% O delimitador desejado vai entre as chaves. No exemplo, não há nada,
% de modo que o delimitador é vazio. Este recurso está sendo usado para
% eliminar o espaço que geralmente existe entre as colunas
\begin{tabularx}{\linewidth}{  }
% A figura foi colocada dentro de um parbox para que fique verticalmente
% centralizada em relação ao resto da linha
\parbox[c]{3cm}{\includegraphics[width=\linewidth]{IFRN}} &
\begin{center}
\textsf{\textsc{UNIVERSIDADE CATÓLICA DE PERNAMBUCO \\
CURSO DE CIÊNCIA DA COMPUTAÇÃO

}}
\end{center}

\end{tabularx}


% O vfill é um espaço vertical que assume a máxima dimensão possível
% Os vfill's desta página foram utilizados para que o texto ocupe
% toda a folha
\vfill

\LARGE

\textbf{
Avaliação de Conformidade em Segurança e Gerenciamento de Eventos com Auxilio da Nuvem
}

\vfill

\Large

\textbf{Fulano}

\vfill

\normalsize

Prof. Rafael Roque

\vfill

\hfill


\vfill

\large



\end{center}

\end{titlepage}
